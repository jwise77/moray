%\documentclass[apj]{emulateapj}
\documentclass[12pt,preprint]{aastex}

\usepackage{graphicx}
\usepackage{epsfig}
\usepackage{natbib}
\usepackage[section] {placeins}
\bibliographystyle{apj}
\citestyle{apj}

%%%%%%%% Begin custom definitions %%%%%%%%%%%%%

\input macros.tex

%%%%%%%% End custom definitions %%%%%%%%%%

\begin{document}

\shorttitle{AMR SIMULATIONS WITH ADAPTIVE RAY TRACING}
\shortauthors{WISE ET AL.}

\title{Enzo+Moray: Radiation Hydrodynamics Adaptive Mesh Refinement
  Simulations with Adaptive Ray Tracing} 

\author{John H. Wise\altaffilmark{1,2}, 
  Tom Abel\altaffilmark{3}, 
  Ji-hoon Kim\altaffilmark{3}, 
  Peng Wang\altaffilmark{3,4},
  Peter Iannucci\altaffilmark{5}}

\altaffiltext{1}{Department of Astrophysical Sciences, Princeton
  University, Peyton Hall, Ivy Lane, Princeton, NJ 08544}
\altaffiltext{2}{Hubble Fellow}
\altaffiltext{3}{Kavli Institute for Particle Astrophysics and
  Cosmology, Stanford University, Menlo Park, CA 94025}
\altaffiltext{4}{NVIDIA}
\altaffiltext{5}{Massachusetts Institute of Technology}
\email{jwise@astro.princeton.edu}

\begin{abstract}

  We describe a photon-conserving numerical method that solves the
  radiative transfer equation, using a spatially-adaptive ray tracing
  scheme, and its parallel implementation into the adaptive mesh
  refinement (AMR) code, Enzo.  Coupling the solver with the energy
  equation and non-equilibrium chemistry network, our radiation
  hydrodynamics framework can be utilized to study a broad range of
  astrophysical problems, such as stellar and black hole (BH)
  feedback.  Inaccuracies can arise from large timesteps and poor
  sampling, therefore we devised an adaptive timestepping scheme and a
  fast approximation of the optically-thin radiation field with
  multiple sources.  We test the method with several radiative
  transfer and radiation hydrodynamics tests that are given in
  \citet{RT06, Iliev09}.  We further test our method with more
  dynamical situations, for example, the propagation of an ionization
  front through a Rayleigh-Taylor instability, time-varying
  luminosities, and collimated radiation.  This method linearly scales
  with the number of point sources and number of grid cells.  To
  combat this, we use a novel method of merging rays at large radius,
  using a tree method, which we briefly describe.  Our implementation
  is scalable to O(10$^3$) processors on distributed and shared memory
  machines and can include radiation pressure and secondary
  ionizations from X-ray radiation.  It is included in the newest
  public release of Enzo.
  
\end{abstract}

\keywords{cosmology --- methods: numerical --- hydrodynamics ---
  radiative transfer}

\section{Introduction}

Radiative transfer is ubiquitous in many astrophysical problems,
encompassing topics such as stellar atmospheres, the interstellar
medium (ISM), galaxy formation, and cosmology.  Thus it is a
well-studied problem \citep[e.g.][and more!]{Rybicki}; however, its
treatment in multi-dimensional calculations is difficult because of
the dependence on seven variables --- three spatial, two angular,
frequency, and time.  The non-local nature of the thermal and
hydrodynamical response to radiation sources further adds to the
difficulty.  In the general case, one must consider all radiation
sources at every point in calculation.  (add: simplify by symmetries
and special cases)

There are several methods to solve the radiative transfer.

\section{Treatment of Radiative Transfer}

Radiation trnasport is a well-studied topic, and we begin by
describing our approach in solving the radiative transfer equation,
which in comoving coordinates \citep{Gnedin97} is
%
\begin{equation}
  \label{eqn:rteqn}
  \frac{1}{c} \; \frac{\partial I_\nu}{\partial t} + 
  \frac{\hat{n} \cdot \nabla I_\nu}{\bar{a}} -
  \frac{H}{c} \; \left( \nu \frac{\partial I_\nu}{\partial \nu} -
  3 I_\nu \right) = -\kappa_\nu I_\nu + j_\nu .
\end{equation}
%
Here $I_\nu \equiv I(\nu, \mathbf{x}, \Omega, t)$ is the radiation
specific intensity in units of energy per time $t$ per solid angle per
unit area per frequency $\nu$.  $H = \dot{a}/a$ is the Hubble
constant, where $a$ is the scale factor.  $\bar{a} = a/a_{em}$ is the
ratio of scale factors at the current time and time of emission.  The
second term represents the propagation of radiation, where the factor
$1/a$ accounts for cosmic expansion.  The third term describes both
the cosmological redshift and dilution of radiation.  On the right
hand side, the first term considers the absorption coefficient
$\kappa_\nu \equiv \kappa_\nu(\mathbf{x},\nu,t)$, and the second term
$j_\nu \equiv j_\nu(\mathbf{x},\nu,t)$ is the emission coefficient
that includes any point sources of radiation or diffuse radiation.

Solving this equation is difficult because of its high dimensionality;
however, we can make some appropriate approximations to reduce its
complexity in order to include radiation transport in numerical
calculations.  Typically timesteps in dynamic calculations are small
enough so that $\Delta a/a \ll 1$, therefore $\bar{a} = 1$ in any
given timestep, reducing the second term to $\hat{n} \partial
I_\nu/\partial \mathbf{x}$.  To determine the importance of the third
term, we evaluate the ratio of the third term ot the second term.
This is $HL/c$, where $L$ is the simulation box length.  If this ratio
is $\ll 1$, we can ignore the third term.  For example at $z=5$, this
ratio is 0.1 when $L = c/H(z=5)$ = 53 proper Mpc.  In large boxes
where the light crossing time is comparable to the Hubble time, then
it could be important to consider cosmological redshifting and
dilution of the radiation, which we will describe later in
\S\ref{sec:ART}.  Thus equation (\ref{eqn:rteqn}) reduces to the
non-cosmological form in this local approximation,
%
\begin{equation}
  \frac{1}{c} \frac{\partial I_\nu}{\partial t} + 
  \hat{n} \frac{\partial I_\nu}{\partial \mathbf{x}} =
  -\kappa_\nu I_\nu + j_\nu .
\end{equation}
%
We choose to represent the source term $j_\nu$ as point sources of
radiation (e.g. stars, quasars) that emit radial rays that are
propaged along the direction $\hat{n}$.  Next we describe this
discretization and its contribution to the radiation field.

\input table1.tex

\subsection{Adaptive Ray Tracing}
\label{sec:ART}

Ray tracing is an accurate method to propagate radiation from point
sources on a computational grid, given that there are rays passing
through each cell.  Along a ray, the radiative transfer equation reads
%
\begin{equation}
\label{eqn:rtray}
\frac{1}{c} \frac{\partial P}{\partial t} + \frac{\partial P}{\partial
  r} = -\kappa P,
\end{equation}
%
where $P$ is the photon number flux along the ray.  To sample the
radiation field at large radii, ray tracing requires at least $N_{ray}
= 4\pi R^2 / (\Delta x)^2$ rays to sample each cell with one ray,
where $R$ is the radius from the source to the cell and $\Delta x$ is
the cell width.  If one were to trace $N_{ray}$ rays out to $R$, the
cells at a smaller radius $r$ would be sampled by, on average,
$(r/R)^2$ rays, which is computationally wasteful because only a few
rays per cell, as we will show later, provides an accurate calculation
of the radiation field.

We avoid this inefficiency by utilizing adaptive ray tracing
\citep{Abel02_RT}, which progressively splits rays when the sampling
becomes too coarse and is based on Hierarchical Equal Area isoLatitude
Pixelation \citep[HEALPix;][]{HEALPix}.  In this scheme, the rays are
traced along normal directions of the centers of HEALPix pixels, which
evenly divides a sphere into equal areas.  The rays are initialized at
each point source with the photon luminosity (ph s$^{-1}$) equally
spread across $N_{\rm pix} = 12 \times 4^l$ rays, where $l$ is the
initial HEALPix level.  We usually find $l$ = 0 or 1 is sufficient
because these coarse rays will be usually be split before traversing
the first cell.

The rays are traced through the grid in a typical fashion
\citep[e.g.][]{Abel99_RT}, in which we calculate the next cell
boundary crossing.  The ray segment length crossing the cell is
%
\begin{equation}
  \label{eqn:trace}
  dr = R_0 - \min_{i=1 \rightarrow 3} \left[(x_{\rm cell,i} - x_{\rm src,i}) /
    \hat{n}_{\rm ray,i} \right],
\end{equation}
%
where $R_0$, $\hat{n}_{\rm ray}$, $x_{\rm cell,i}$, and $x_{\rm
  src,i}$ are the initial distance travelled by the ray, normal
direction of the ray, the next cell boundary crossing on the $i$-th
dimension, and the position of the point source that emitted the ray,
respectively.  However before the ray travels across the cell, we
evaluate the ratio of the face area $A_{\rm cell}$ of the current cell
and the solid angle $\Omega_{\rm ray}$ of the ray,
%
\begin{equation}
  \label{eqn:split}
  \Phi_c = \frac{A_{\rm cell}} {\Omega_{\rm ray}} = 
  \frac{N_{\rm pix} (\Delta x)^2} {4\pi R_0^2}.
\end{equation}
%
If $\Phi_c$ is less than a pre-determined value (usually $>2$), the
ray is split into 4 child rays.  We investigate the variations in
solutions with $\Phi_c$ in \S\ref{sec:ang_dep}.  The pixel numbers
of the child rays $p^\prime$ are given by the ``nested'' scheme of
HEALPix at the next level, i.e. $p^\prime = 4 \times p + [0,1,2,3]$,
where $p$ is the original pixel number.  The child rays (1) acquire
the new normal vectors of the pixels, (2) retain the same radius of
the parent ray, and (3) gets a quarter of the photon flux of the
parent ray.  Afterwards the parent ray is discontinued.

A ray propagates and splits until 
%
\begin{enumerate}
\item the photon has travelled $c \times dt_P$, where $dt_P$ is the
  radiative transfer timestep,
\item its photon flux is almost fully absorbed ($>99.9\%$) in a single
  cell, which significantly reduces the computational time if the
  radiation volume filling fraction is small,
\item the photon leaves the computational domain with isolated
  boundary conditions, or
\item the photon travels $\sqrt{3}$ of the simulation box length with
  periodic boundary conditions.
\end{enumerate}
%
In the second case, the photon is halted at that position and saved,
where it will be considered in the solution of $I_\nu$ at the next
timestep.  In the next timestep, the photon will encounter a different
hydrodynamical and ionization state, hence $\kappa$, in its path.
Furthermore any time variations of the luminosities will be retained
in the radiation field.  In this sense, our method retains the time
derivative of the radiative transfer equation.

\subsection{Ray Merging}

Unfortunately the computational work of ray tracing scales with the
number of sources.  To resolve this barrier, rays can be merged in
cases with either nearly parallel rays from clustered sources of
radiation, whether it be diffuse radiation or point sources, or where
rays exit a region with high resolution into a coarsely resolved
region.  In each case, the factor $\Phi_c$ can be much higher than the
desired value, and it would be efficient to merge these rays.  We have
devised a new scheme that merges rays from clustered sources based on
a hierarchical binary tree, which is analogous to the gravity tree
solvers.  We show the performance improvements in the strong scaling
tests (\S\ref{sec:strong_sc}).  However we leave its full details for
a later paper.

\subsection{Radiation Field}

The radiation field is calculated by integrating equation
\ref{eqn:rtray} along each ray, which is done by considering the
discretization of the ray into segments.  In the following section, we
assume the rays are monochromatic.  For convenience, we express the
integration in terms of optical depth $\tau = \int \kappa(r,t) \;
dr$, and for a ray segment,
%
\begin{equation}
  \label{eqn:dtau}
  d\tau = \sigma_{\rm abs}(\nu) n_{\rm abs} dr.
\end{equation}
Here $\sigma_{\rm abs}$ and $n_{\rm abs}$ are the cross section and
number density of the absorbing medium, respectively.  We use the
cell-centered density in our calculations but have experimented with
trilinearly interpolated densities \citep[see][]{Mellema06} without
producing considerable artifacts.  Equation (\ref{eqn:rtray}) has a
simple exponential analytic solution, and the photon flux of a ray is
reduced by
%
\begin{equation}
  \label{eqn:flux}
  dP = P \times (1 - e^{-\tau})
\end{equation}
as it crosses a cell.  We equate the photo-ionization rate to the
absorption rate, resulting in photon conservation \citep{Abel99_RT,
  Mellema06}.  Thus the photo-ionization $k_{\rm ph}$ and
photo-heating $\Gamma_{\rm ph}$ rates associated with a single ray are
%
\begin{equation}
  \label{eqn:kph}
  k_{\rm ph} = \frac{P (1 - e^{-\tau})}{n_{\rm abs} \; V_{\rm cell} \; dt_P},
\end{equation}
\begin{equation}
  \label{eqn:gamma}
  \Gamma_{\rm ph} = k_{\rm ph} \; (E_{\rm ph} - E_i),
\end{equation}
where $V_{\rm cell}$ is the cell volume, $E_{\rm ph}$ is the photon
energy, and $E_i$ is the ionization energy of the absorbing material.
In each cell, the photo-ionization and photo-heating rates from each
ray in the calculation are summed, and after the ray tracing is
complete, these rates can be used to update the ionization state and
energy of the cells.  Considering a system with only hydrogen
photo-ionizations and radiative recombinations, these changes are very
straightforward and is useful for illustrative purposes.  The change
in neutral hydrogen is
%
\begin{equation}
\frac{dn_{\rm H}}{dt} = \alpha_B n_e n_p - C_{\rm H} n_e n_{\rm
  H} - k_{\rm ph},
\end{equation}
where $\alpha_B = 2.59 \times 10^{-13} \mathrm{cm}^3 \mathrm{s}^{-1}$
is the recombination coefficent at 10$^4$ K in the Case B on-the-spot
approximation, in which all recombinations are locally reabsorbed,
\citep{Spitzer78}, and $C_{\rm H}$ is the collisional ionization rate.
However for more accurate solutions in calculations that consider
several chemical species, the photo-ionization rates are better
utilized in solvers that consider chemical networks
\citep[e.g.][]{Abel97}.

\subsection{Additional Physics}
\label{sec:addphysics}

Other radiative processes can also be important in some situations,
such as attentuation of radiation in the Lyman-Werner bands, secondary
ionizations from X-ray radiation, Compton heating of from scattered
photons, and radiation pressure.

\subsubsection{Absorption of Lyman-Werner Radiation}

Molecular hydrogen can absorb photons in the Lyman-Werner bands that
are composed of 76 absoprtion lines ranging from 10.2 to 13.6 eV.
Each of these spectral lines can be modelled with a typical
exponential attenuation equation \citep{Ricotti01}, but
\citet{Draine96} showed that this self-shielding is well modeled with
the following relation to total \hh~column density
%
\begin{equation}
  \label{eqn:lwband}
  f_{\rm shield}(N_{\rm H2}) = \left\{ \begin{array}{l@{\quad}l}
      1 & (N_{\rm H2} \le 10^{14} \; \mathrm{cm}^{-2})\\
      (N_{\rm H2}/10^{14} \; \mathrm{cm}^{-2})^{-0.75} & (N_{\rm H2} >
        10^{14} \; \mathrm{cm}^{-2})
    \end{array} \right. .
\end{equation}
To incorporate this shielding function into the ray tracer, we store
the total \hh~column density and calculate the \hh~dissociation rate
by summing the contribution of all rays,
%
\begin{equation}
  \label{eqn:lwRT}
  k_{\rm diss} = \sum_{\rm rays} \frac{P \; \sigma_{\rm LW} \;
    \Omega_{\rm ray} \; r^2 \; dr}{A_{\rm cell} \; dV \; dt_P},
\end{equation}
where $\sigma_{\rm LW} = 3.71 \times 10^{18} \mathrm{cm}^2$ is the
effective cross-section of \hh~\citep{Abel97}.  To account for any
absorption in each photon package, we attentuate the photon number
flux by
%
\begin{equation}
  \label{eqn:LWdP}
  dP = P [f_{\rm shield}(N_{\rm H2} + dN_{\rm H2}) - f_{\rm shield}(N_{\rm H2})],
\end{equation}
where $dN_{\rm H2}$ is the \hh~column density in the current cell.

\subsubsection{Secondary Ionizations from X-rays}
\label{sec:xrays}

A high-energy ($E_{\rm ph} \gsim 100$ eV) photon can ionize multiple
neutral hydrogen and helium atoms, and this should be considered in
such radiation fields.  \citet{Shull85} studied this effect with Monte
Carlo calculations over varying electron fractions and photon energies
up to 3 keV.  They find that the excitation of hydrogen and helium and
the ionization of \ion{He}{2} is negligible.  The number of secondary
ionizations of H and He is reduced from the ratio of the photon and
ionization energies ($E_{\rm ph} / E_i$) by a factor of
%
\begin{equation}
  Y_{\rm k,H} = 0.3908 (1 - x^{0.4092})^{1.7592},
\end{equation}
\begin{equation}
  Y_{\rm k,He} = 0.0554 (1 - x^{0.4614})^{1.6660},
\end{equation}
where $x$ is the electron fraction.  The remainder of the photon
energy is deposited into thermal energy that is approximated by
%
\begin{equation}
  Y_{\rm \Gamma} = 0.9971 [ 1 - (1 - x^{0.2663})^{1.3163} ]
\end{equation}
and approaches one as $x \rightarrow 1$.  Thus in gas with low
electron fractions, most of the energy results in ionizations of
hydrogen and helium, and in nearly ionized gas, the energy goes into
photo-heating.

\subsubsection{Compton Heating from Photon Scattering}

High energy photons can also cause Compton heating by scattering off
free electrons.  During a scattering, a photon loses $\Delta E(T_e) =
4kT_e \times (E_{ph} / m_e c^2)$ of energy, where $T_e$ is the
electron temperature.  For the case of monochromatic energy groups, we
model this process by considering that the photons are absorbed by a
factor of 
\begin{equation}
  \frac{dP_C}{P} = (1 - e^{-\tau_e}) \frac{\Delta(T_e)}{E_{ph}},
\end{equation}
which is the equivalent of the photon energy decreasing.  Here $\tau_e
= n_e \sigma_{KN} dl$ is the optical depth to Compton scattering, and
$\sigma_{KN}$ is the non-relativistic Klein-Nishina cross section
\citep{Rybicki}.  The Compton heating rate is thus
\begin{equation}
  \Gamma_{\rm ph,C} = \frac{dP_C}{n_e \; V_{\rm cell} \; dt}.
\end{equation}

\subsubsection{Radiation Pressure}

The absorption of radiation transfers momentum from photons to the
absorbing medium, i.e. radiation pressure.  This is easily computed by
considering the momentum
\begin{equation}
  d\mathbf{p}_\gamma = \frac{dP \; E_{\rm ph}}{c\;dt} \; \hat{r}
\end{equation}
of the absorbed radiation from the incoming ray, where $\hat{r}$ is
the normal direction of the ray and $dt$ is the radiative transfer
timestep.  The resulting acceleration of the gas due to radiation
pressure is
\begin{equation}
  d\mathbf{a} = d\mathbf{p}_\gamma / (\rho \; V_{\rm cell}),
\end{equation}
where $\rho$ is the gas density inside the cell.  This acceleration is
then added to the other forces, e.g. gravity and gas pressure, in the
calculation.

\subsection{Geometric Corrections}
\label{sec:meth_fc}

\begin{figure}[t]
  \plottwo{fig/covering.eps}{fig/calc_fc.eps}
  \caption{\label{fig:covering} (a) A two-dimensional illustration of
    the overlap between the beam associated with a ray $\gamma$ and a
    computational cell.  The ray has a segment length of $dr$ passing
    through the cell.  The covering area is denoted by dark grey,
    where the full area $(dr \times L_{\rm pix})$ is colored by the
    dark and light grey.  The photo-ionization and photo-heating rates
    should be corrected by this overlap fraction $f_c$.  (b)
    Annotation of quantities used in this geometric correction.}
\end{figure}

Consider the solid angle $\Omega_{\rm ray}$ and photon flux $P$
associated with a single ray, and assume the flux is constant across
$\Omega_{\rm ray}$.  There exists a discrenpancy between the geometry
cell face and HEALPix pixel when the pixel does not cover the entire
cell face, which is illustrated in Figure \ref{fig:covering}.  This
mismatch causes non-spherical artifacts and is most apparent in the
optically thin case, where the area of the pixel is dominant over $(1
- e^\tau)$ when calculating $k_{\rm ph}$.  One can avoid these
artifacts by increasing the sampling $\Phi_c$ to high values,
e.g. $>10$, but we have formulated a simple geometric correction to
the calculation of the radiation field.  This correction is not unique
to the HEALPix formalism but can be applied to any type of
pixelization.

The contribution to $k_{\rm ph}$ and $\Gamma_{\rm ph}$ must be
corrected by a covering factor $f_c$.  When the pixel is fully
contained within the cell face, $f_c \equiv 1$.  Because the geometry
of the pixel can be complex, i.e. curved edges, we approximate $f_c$
by assuming the pixel is square.  The covering factor is thus related
to the width of a pixel, $L_{\rm pix} = R_0 \, \theta_{\rm pix}$, and
the distance from the ray segment midpoint to the closest cell
boundary, which is depicted in Figure \ref{fig:covering}.  To estimate
$f_c$, we first find the distance $d_{\rm center,i}$ from the midpoint
of the ray segment to the cell center $x_{\rm 0,i}$ in orthogonal
directions,
%
\begin{equation}
  \label{eqn:midpoint}
  D_{\rm c,i} = \left| R_{\rm 0,i} + \hat{n}_i\,\frac{dr}{2} - x_{\rm
      0,i} \right|.
\end{equation}
The distance to the closest cell boundary is $D_{\rm edge} = dx/2 -
\min_{i=1 \rightarrow 3} (D_{\rm c,i})$.  Thus the covering
factor is related to the square of the ratio between the $L_{\rm pix}$
and $d_{\rm mid}$ by
%
\begin{equation}
  \label{eqn:fc}
  f_c = \left( \frac{1}{2} + \frac{D_{\rm edge}}{L_{\rm pix}} \right)^2
\end{equation}
One half of the pixel is always contained within the cell, which
results in the factor of 1/2.  Finally we multiply $k_{\rm ph}$ and
$\Gamma_{\rm ph}$ by $f_c$ but leave the absorbed radiation $dP$
untouched because this would underestimate the attenuation of the
incoming radiation.  Using $f_c$ calculated like above, the method is
no longer photon conserving.  In our implementation, we felt that the
spherical symmetry obtained outweighed the loss of photon
conservation.  We show the deviations from fully conservative in
\S\ref{sec:test_fc}.  

We briefly next describe how to retain photon conservation with a
geometric correction.  Notice that we compute $f_c$ by only
considering the distances in orthogonal directions.  A better estimate
would consider the distance between the cell boundary and ray segment
midpoint in the direction of $\mathbf{x}_{\rm mid} - \mathbf{x}_{\rm
  center}$.  However we find that the method outlined here provides a
sufficient correction factor to avoid any non-spherical artifacts.
Furthermore in principle, the ray should also contribute to any
neighboring cells that overlap with $\Omega_{\rm ray}$, which is the
key to be photon conservative with such a geometric correction.

\subsection{Optically Thin Approximation}

When the medium is optically thin to the radiation, radiation is only
attentuated by geometric dilution in the local approximation to
Eq. (\ref{eqn:rteqn}), i.e. the inverse square law.  We can minimize
the computational work of ray tracing in the optically thin regime by
utilizing this fact.  Here we track the total column density $N_{\rm
  abs}$ and the equivalent total optical depth $\tau$ traversed by the
photon.  If $\tau < \tau_{\rm thin} \sim 0.1$ after the ray exits the
cell, we calculate the photo-ionization and photo-heating rates
directly from the incoming ray instead of the luminosity of the
source.  This should only be evaluated once per cell per radiation
source.
\begin{equation}
  \label{eqn:optthin}
  k_{\rm ph} = \frac{\sigma_{\rm abs} \; P}{dt_P \; \theta_{\rm pix}}
  \; \frac{r_{\rm cell}}{r_{\rm ray}}.
\end{equation}
Note that the photon number $P$ in the ray has already been
geometrically diluted by ray splitting.  Here $r_{\rm cell}$ and
$r_{\rm ray}$ are the radii from the radiation source to the cell
center and where the ray exits the cell.  Thus the last factor
corrects the flux to a value appropriate for the cell center.  The
photo-heating ray is calculated in the same manner as the general
case, $\Gamma_{\rm ph} = k_{\rm ph} (E_{\rm ph} - E_i)$.  No photons
are removed from the ray.  With this method, we only require one ray
travel through each cell where the gas is optically thin, thus
reducing the computational expense.

We must be careful not the overestimate the radiation when multiple
rays enter a single cell.  In the case of a single radiation source,
the solution is simple --- only assign the cell the photo-ionization
and photo-heating rates when $k_{\rm ph} = 0$.  However in the case
with multiple sources, this is no longer valid, and we must sum the
flux from all optically thin sources.  Only one ray per source must
contribute to a single cell in this framework.  We create a flagging
field that marks whether a cell has already been touched by an
optically thin photon from a particular radiation source.  Naively, we
would be restricted to tracing rays from a single source at a time if
we use a boolean flagging field.  However we can trace rays for 32
sources at a time by using bitwise operations on a 32-bit integer
field.  For example in \texttt{C}, we would check if an optically thin
ray from the \texttt{n}-th source has propagated through cell
\texttt{i} by evaluating \texttt{(MarkerField[i] $\gg$ n \& 1)}.  If
false, then we can add the optically thin approximation
(Eq. \ref{eqn:optthin}) to the cell and set \texttt{MarkerField[i] |=
  (1 $\gg$ n);} to mark the cell.

\section{Numerical Implementation in Enzo}

\enzo~is a parallel block-structured AMR \citep{BergerAMR} code that
is publicly available\footnote{http://lca.ucsd.edu/enzo}
\citep{BryanNorman1997, OShea2004}.  In this section, we describe our
parallel implementation of the adaptive ray tracing method into \enzo.
First we explain the programming design of handling the ``photon
packages'' that are traced along the adaptive rays.  We use the terms
photon packages and rays interchangably.  Next we focus on the details
of the radiation hydrodynamics and then the importance of correct
timestepping.  Last we give our parallelization strategy of tracing
rays through an AMR hierarchy.  This implementation is included in the
\texttt{v2.0} public version of \enzo.

\subsection{Programming Design}
\label{sec:design}

Each photon package is stored in the AMR grid with the finest
resolution that contains its current position.  The photon packages
keep track of their (1) photon flux, (2) photon type, (3) photon
energy, (4) the length of its emission, (5) emission time, (6) current
time, (7) radius, (8) total column density, (9) HEALPix pixel number,
(10) HEALPix level, and (11) position of the originating source,
totaling 60 (88) bytes for single (double) precision.  When \enzo~uses
double precision for grid and particle positions and time, items 4-7
and 11 are double precision.

We only treat point sources of radiation in our implementation;
therefore all base level photon packages originate from them.  As they
travel away from the source, they generally pass through many AMR
grids, especially if the simulation has a high dynamic range.  This is
a challenging programming task as rays are constantly entering and
exiting grids.  Before any computation, the number of rays in a
particular grid is highly unpredictable because the intervening medium
is unknown.  Furthermore, the splitting of parent rays into child rays
and a dynamic AMR hierarchy add to the complexity.  Because of this,
we store the photon packages as a doubly linked list
\citep{Abel02_RT}.  Thus we can freely add and remove them from grids
without the concern of allocating enough memory before the tracing
commences.

\begin{figure}[t]
  \plotone{fig/RTcoupling.eps}
  \caption{\label{fig:MainAlgorithm} Flow chart for the overall
    algorithm of the radiative transfer module in \enzo~that
    illustrates (1) the creation of photon packages, (2) ray tracing,
    (3) the transport of photon packages between AMR grids, and (4)
    coupling with the hydrodynamics.  The ray tracing algorithm, which
    is contained in the ``Trace Rays'' is detailed in Figure
    \ref{fig:RTalgorithm}.}
\end{figure}

\begin{figure}[t]
  \plotone{fig/RTalgorithm.eps}
  \caption{\label{fig:RTalgorithm} Flow chart for the ray tracing
    algorithm for one photon passing through a grid.  Note that only
    one step is needed in the routine to adaptively split rays.  The
    remainder is a typical ray tracing method.}
\end{figure}

We illustrate the underlying algorithm of the ray tracing module in
\enzo~in Figure \ref{fig:MainAlgorithm} and the ray tracing algorithm
is shown in Figure \ref{fig:RTalgorithm}.  The module is only called
when advancing the finest AMR level.  We describe its steps below.

\step{1} Create $N_{\rm pix}$ new photon packages on the initial
HEALPix level from point sources.  Place the new rays in the highest
resolution AMR grid that contains the source.

\step{2} Initialize all radiation fields to zero.

\step{3} Loop through all AMR grids, tracing any rays that exist in
it.  For each ray, the following substeps are taken.

\step{3a} Compute the ray normal based on the HEALPix level and pixel
number of the photon package with the HEALpix routine
\texttt{pix2vec\_nest}.  One strategy to accelerate the computation is
to store ray segment paths in memory \citep{Abel02_RT,
  Krumholz07_ART}; however this must be recomputed if the grid
structure or point source position changes.  We do not restrict these
two aspects and cannot employ this acceleration method.

\step{3b} Compute the position of the ray ($\mathbf{r}_{\rm src} + r
\mathbf{\hat{n}}$), the current cell coordinates in floating point and
its corresponding integer indices.  Here $\mathbf{r}_{\rm src}$ is the
position of the point source, $r$ is the distance travelled by the
ray, and $\mathbf{\hat{n}}$ is the ray normal.

\step{3c} Check if a subgrid exists under the current cell.  If so,
move the ray to a linked list that contains all rays that should be
moved to other grids.  We call this variable \texttt{PhotonMoveList}.
Store the destination grid number and level.  Continue to the next ray
in the grid (step 3a).  We determine whether a subgrid exists by
creating a temporary 3D field of pointers that either equals the
pointer of the current grid if no subgrid exists under the current
cell or the child pointer that exists under the current cell.  This
provides a significant speedup when compared to a simple comparision
of a photon package position and all of the child grid boundaries.
Note that this is the same algorithm used in \enzo~when moving
collisionless particles to child grids.

\step{3d} Compute the next cell crossing of the ray and the ray
segment length across the cell (Equation \ref{eqn:trace}).

\step{3e} Compare the solid angle associated with the ray at radius
$r+dr$ with a user-defined splitting criterion (Equation
\ref{eqn:split}).  If the solid angle is larger than the desired
minimum sampling, split the ray into 4 child rays (\S\ref{sec:ART}).
These rays are inserted into the linked list after the parent ray,
which is subsequently deleted.  Continue to the next ray (step 3a),
which will be the first child ray.

\step{3f} Calculate the geometric correction (Equation \ref{eqn:fc}),
the optical depth of the current cell (Equation \ref{eqn:dtau}),
photo-ionization and photo-heating rates (Equations \ref{eqn:kph} and
\ref{eqn:gamma}), and add the column density of the cell to the total
column density of the ray.

\step{3g} Add the effects of any optional physics modules
(\S\ref{sec:addphysics})---secondary ionizations from X-rays, Compton
heating from scattering, and radiation pressure.

\step{3h} Update the current time ($t = t + cdr$), photon flux ($P = P
- dP$, Equation \ref{eqn:flux}), and radius of the ray ($r = r + dr$).

\step{3i} If the photon flux is zero or the total optical depth is
large ($>20$), delete the ray.

\step{3j} Check if the ray has travelled a total distance of $c dt_P$
in the last timestep.  If we are keeping the time-derivative of the
radiative transfer equation, halt the photon.  If not (i.e. infinite
speed of light), delete the photon.

\step{3k} Check if the ray has exited the current grid.  If false,
continue to the next ray (step 3a).  If true, move the ray to the
linked list \texttt{PhotonMoveList}, similar to step 3c.  If the ray
exits the simulation domain, delete it if the boundary conditions are
isolated; otherwise, we change the source position of the ray by a
distance \texttt{-sign(n[i]) * DomainWidth[i]}, where \texttt{n} is
the ray normal, and \texttt{i} is the dimension of the outer boundary
it has crossed.  The radius is kept unchanged.  In essence, this
creates a ``virtual source'' outside the box because the ray will be
moved to the opposite side of the domain, appearing that it originated
from this virtual source.

\step{4} If any rays exist in the linked list \texttt{PhotonMoveList},
move them to their destination grids and return to step 3.  This
requires MPI communication if the destination grid exists on another
processor.

\step{5} If all rays have not been halted (keeping the time-derivative
of the radiative transfer equation), absorbed, or exited the domain,
return to step 3.

\step{6} With the radiation fields updated, call the chemistry and
energy solver and update only the cells with radiation, which is
discussed further in \S\ref{sec:coupling}.

\step{7} Advance the time associated with the photons $t_P$ by the
global timestep $dt_P$ (for its calculation, see
\S\ref{sec:timestepping}).  If $t_P$ does not exceed the time on the
finest AMR level, return to step 1.

\subsection{Energy groups}

In our implementation, photon packages are mono-chromatic and are
assigned a photon type that corresponds whether it is a photon that
(1) ionizes hydrogen, (2) singly ionizes helium, (3) doubly ionizes
helium, (4) has an X-ray energy, or (5) dissociates molecular hydrogen
(Lyman-Werner radiation).  One disadvantage of mono-chromatic rays is
the number of rays increase with the number of frequency bins.
However this allows for early termination of rays that are fully
absorbed, which are likely to have high absorption cross-sections
(e.g. \ion{H}{1} ionizations near 13.6 eV) or a low initial intensity
(e.g. \ion{He}{2} ionizing photons in typical stellar populations).
The other approach used by some groups \citep[e.g.][]{Trac07} is to
store all energy groups in a single ray.  This reduces the number of
the rays generated and the computation associated with the ray
tracing.  Unless the ray dynamically adjusts its memory allocation for
the energy groups as they become depleted, this method is also memory
intensive in the situation where most of the energy groups are
completely absorbed but a few groups still have significant flux.

In practice, we have found that one energy group per photon type is
sufficent to match expected analytical tests.  For example when
modeling Population III stellar radiation \citep[e.g.][for hydrogen
  ionizing radiation only]{Abel07, Wise08b}, we have 3 energy
groups---\ion{H}{1}, \ion{He}{1}, \ion{He}{2}---each with an energy
that equals the average photon energy above the ionization threshold.

\subsection{Coupling with Hydrodynamics}
\label{sec:coupling}

Solving the radiative transfer equation is already an intensive task,
but coupling the effects of radiation to the gas dynamics is even more
difficult because the radiation fields must be updated on a timescale
such that it can react to the radiative heating, i.e. sound-crossing
time.  The frequency of its evaulation will be discussed in the next
section.

\enzo~solves the physical equations in an operator-split fashion over
a loop of AMR grids.  On the finest AMR level, we call our radiation
transport solver before this main grid loop in the following sequence:
\begin{itemize}
\item \textbf{All grids:}
  \begin{enumerate}
  \item Solve for the radiation field with the adaptive ray tracer
  \item Update species fractions and energies for cells with radiation
    with a non-equilibrium chemistry solver on subcycles (Equation
    \ref{eqn:rate_dt}).
  \end{enumerate}
\item \textbf{For each grid:}
  \begin{enumerate}
  \item Solve for the gravitational potential with the particle mesh
    method
  \item Solve hydrodynamics
  \item Update species fractions and energies for cells without
    radiation with a non-equilibrium chemistry solver on subcycles
    (Equation \ref{eqn:rate_dt}).
  \item Update particle positions
  \item Star particle formation
  \end{enumerate}
\item \textbf{All grids:} Update solution from children grids
\end{itemize}
%

Since the solver must be called many times, the efficiency of the
radiation solver is paramount.  After every radiation timestep, we
call the non-equilibrium chemistry and energy solver in \enzo.  This
solves both the energy equation and the network of stiff chemical
equations on small timesteps, i.e. subcycles \citep{Anninos97}.  The
timestep is
%
\begin{equation}
  \label{eqn:rate_dt}
  dt = \min\left(
    \frac{0.1n_e}{|dn_e/dt|}, 
    \frac{0.1n_{\rm HI}}{|dn_{\rm HI}/dt|}, 
    \frac{0.1e}{|de/dt|}, 
    \frac{dt_{\rm hydro}}{2}\right),
\end{equation}
where $n_e$ is the electron number density, $e$ is the specific
energy, and $dt_{\rm hydro}$ is the hydrodynamic timestep.  This
limits the subcycle timestep to a 10\% change in either electron
density, neutral hydrogen density, or specific energy.  In simulations
without radiation, \enzo~calls this solver in a operation-split manner
after the hydrodynamics module for grids only on the AMR level that is
being solved.  In simulations with radiative transfer, the radiation
field can change on much faster timescales than the normal
hydrodynamical timesteps.

For example, a grid on level $L$ might have no radiation in its
initial evaluation, but the ionization front exists just outside its
boundary.  Then radiation permeates the grid in the time between
$t_{\rm L=1} \rightarrow t_{\rm L=1}+dt_{\rm L=1}$, and the energy and
chemical state of the cells must be updated again to advance the
ionization front accurately.  If one does not update these cells, it
will appear that the ionization front does not enter the grid until
the next hydrodynamical timestep!  Visually this appears as
discontinuities in the temperature and electron fraction on grid
boundaries.  One may avert this problem by solving the chemistry and
energy equations for every cell on every radiative transfer timestep,
but this is very time consuming and unnecessary, especially if the
radiation filling factor is small.

We choose to dynamically split the problem by cells with and without
radiation.  In every radiation timestep, the chemo-thermal state of
\textit{only} the cells with radiation are updated.  For the solver
subcycling, we replace $dt_{\rm hydro}$ with $dt_P$ in Equation
\ref{eqn:rate_dt} in this case.  Once the radiative transfer solver is
finished with its timesteps, the hydrodynamic module is called, and
then the chemo-thermal state of the cells without radiation are
updated on a subcycle timestep stated in Equation \ref{eqn:rate_dt}.

For cells that transition from zero to non-zero photo-ionization
rates, the initial state that enters into the chemistry and energy
solver does not correspond to the current time of the radiation
transport solver $t_{\rm RT}$, but either time $t_{\rm L}$ if the grid
level is the finest level because its chemo-thermal state has not been
updated or time $t_{\rm L}+dt_{\rm L}$ on all other levels.  In
principle, one could first revert the cell back to time $t_{\rm L}$
and then update to $t_{\rm RT}$ with the chemistry and energy solver
if the cell is on the finest level.  However in practice, the
timescales in gas without radiation are small compared to the
ionization and heating timescales when radiation is introduced.
Therefore, we do not perform this correction and find that this does
not introduce any inaccuracies in both test problems (see
\S\ref{sec:rt_tests}) and real world applications.


\subsection{Temporal evolution}
\label{sec:timestepping}

There have been several methods of choosing a timestep to solve
radiation transfer equation because an accurate yet large timestep is
not trivial to compute.  We describe several methods to calculate the
radiative transfer timestep in this section.  With a small enough
timestep, the solution is guaranteed to avoid any inaccuracies
(ignoring any systematic ones) due to timestepping, but the solver
must be called many times.  These frequent calls may be unneccesary
because the same solution may be accomplished with a longer timestep.
Furthermore with ray tracing, the photon packages only advance a short
distance, and they will exist in every $dx/dt_P$ cells with radiation
and are stored between timesteps, excessively consuming memory.  On
the shortest timescale, one can safely set the timestep to the
light-crossing time of a cell \citep{Abel99_RT, Trac07} but encounters
the problems stated above.

If the timestep is too large, the solution will become inaccurate;
specifically, ionization fronts will advance too slowly, as radiation
intensity exponentially drops with a scale length of the mean free
path
\begin{equation}
  \label{eqn:mfp}
  \lambda_{\rm mfp} = \frac{1}{n_{\rm abs} \; \sigma_{\rm abs}}
\end{equation}
past the ionization front.  For example in our implementation, the
chemo-thermal state of the system remains constant as the rays are
traced through the cells.  Thus in the case of a single \ion{H}{2}
region, the speed of the ionization front is limited to approximately
the $\lambda_{\rm mfp} / dt_P$.

\subsubsection{Minimizing neutral fraction change}
\label{sec:dt_hi}

\begin{figure}[t]
  \plottwo{fig/dtp_test3noavg.eps}{fig/dtp_test3.eps}
  \caption{\label{fig:dtsmooth} Radiative transfer adaptive timestep
    in shadowing test (Test 3; \S\ref{sec:test3}) while restricting
    the neutral fraction change to 5\% in the ionization front.  The
    unmodified timestep (left) is slightly more noisy and the minima
    are more prominent than the timestep computed with a running
    average of the last two timesteps (right).  The points show every
    tenth timestep taken into account for the running average.  The
    sawtooth behavior is created by the ionization front advancing
    into the next neutral cell in the overdensity.}
\end{figure}

Another strategy is only allowing the neutral fraction of the ionized
gas to change a small amount, i.e. for a single cell,
%
\begin{equation}
  \label{eqn:dtelec}
  dt_{\rm P,cell} = \epsilon_{\rm ion} \frac{n_{\rm HI}}{|dn_{\rm HI}/dt|} =
  \frac{\epsilon_{\rm ion}}{|k_{\rm ph} + n_e (C_{\rm H} + \alpha_B)|},
\end{equation}
where $\epsilon_{\rm ion}$ is the maximum fraction change in neutral
fraction.  \citet{Shapiro04} found that this limited the speed of the
ionization front.  We can investigate this further by evaluating the
ionization front velocity in a growing Str\"{o}mgren sphere without
recombinations, where $\dot{N}_\gamma = 4 \pi R^2 n_{\rm H} v_{\rm
  IF}$.  Using $\kph \propto \dot{n}_{\rm H} / n_{\rm H}$ and $\kph =
\dot{N}_\gamma \sigma / 4\pi R^2 A_{\rm cell}$, we can make
substitutions on both sides of the equation to arrive at the
ionization front velocity $v_{\rm IF} \propto n_{\rm H} /\dot{n}_{\rm
  H}$.  

We have implemented this method but we only consider cells within the
ionization front (by experiment we choose $\tau > 0.5$) because we are
interested in evolving ionization fronts at the correct speed.  In the
ionized region, the absolute changes in neutral fraction are small and
will not significantly affect the ionization front evolution.  In
other words, $n_{\rm HI}/(dn_{\rm HI}/dt)$ may be large but $(dn_{\rm
  HI}/dt) \sim 0$, thus we can safely ignore these cells when
determining the timestep without sacrificing accuracy.

We search for the cell with the smallest $dt_P$ based on Equation
\ref{eqn:dtelec}.  In principle, one could use this value without
modifications as the timestep, but there is considerable noise both
spatially and temporally.  In order to make this technique stable, we
first spatially smooth the values of $dt_{\rm P,cell}$ with a Gaussian
filter over a $3^3$ cube.  Because we only consider the cells within
the ionization front, we set $dt_{\rm P,cell}$ to the hydrodynamical
timestep outside the front during the smoothing.  After we have
smoothed $dt_{\rm P,cell}$, we select the minimum value as $dt_P$.
Significant noise in $dt_P$ can exist between time discretizations.
Because the solution can become inaccurate if the timestep is allowed
to be too large, we restrict the next timestep to be less than twice
the previous timestep,
%
\begin{equation}
  \label{eqn:dtchange}
  dt_{\rm P,1} = \min(2 dt_{\rm P,0},\; dt_{\rm P,1}).
\end{equation}
Otherwise, we still restrict the change in timestep by setting the
next timestep to be the average of the previous timestep $dt_{\rm
  P,0}$ and $\min(dt_{\rm P,cell})$.  Figure \ref{fig:dtsmooth} shows
the smooth evolution of $dt_{\rm P}$ in a growing Str\"{o}mgren sphere when
compared to the values of $\min(dt_{\rm P,cell})$.

\subsubsection{Time averaged quantites within a timestep}

\citet{Mellema06} devised an iterative scheme that allows for large
timesteps while retaining accuracy by considering the time-averaged
values ($\tau$, $k_{\rm ph}$, $n_e$, $n_{\rm HI}$) during the
timestep.  Starting with the cells closest to the source, they first
calculate the column density to the cell.  Then they compute the
time-averaged neutral density for the cell and its associated optical
depth, which is added to the total time-averaged optical depth.  With
these quantities, one can compute a photo-ionization rate and update
the electron density.  This process is repeated until convergence is
found in the neutral number density.  In a test with a Str\"{o}mgren
sphere, they found analytical agreement with $10^{-3}$ less timesteps
than a method without time-averaging.  Another advantage of this
method is the use of pre-calculated tables for the photo-ionization
rates as a function of optical depth, based on a given spectrum.  This
minimizes the energy groups needed to accurately sample a spectrum.
We are currently implementing this method into Enzo+Moray.

\subsubsection{Physically motivated}
\label{sec:dt_const}

In our implementation, a constant timestep is necessary when solving
the time-dependent radiative transfer equation.  It should be small
enough to evolve ionization fronts accurately, as discussed earlier.
The timestep can be based on physical arguments, for example, the
sound-crossing time of an ionized region at $T > 10^4$ K.  To be
conservative, one may choose the sound-crossing time of a cell
\citep[e.g.][]{Abel07, Wise08b}.  Alternatively, the diameter of the
smallest relevant system (e.g., an accretion radius, transition radius
to a D-type ionization front, etc.) in the simulation may be chosen to
calculate the sound-crossing time.

If the timestep is too large, the ionization front will propagate too
slowly, but it evenutually approaches the correct radius at late times
(see \S\ref{sec:dt_dep}).  This does not prevent one from using a
large timestep, particularly if the system is not critically affected
by a slower I-front velocity.  One example is an expanding \ion{H}{2}
region in a power-law density gradient.  After a brief, initial R-type
phase, the I-front becomes D-type phase, where the ionization and
shock front progress jointly at the sound speed of the ionized region.
A moderately large (\textbf{show how large!}) timestep can accurately
follow its evolution.  However after the I-front passes a critical
radius \citep[show this!][]{Franco90}, the I-front detaches from the
shock front and accelerates.  The I-front velocities in these two
stages differ up to a factor of $\sim100$ (\textbf{check this!}).
Although the solution is accurate with a large timestep in the D-type
phase, the I-front may lag behind because of the constant timestep.
After a few recombination times, the numerical solution evenutally
approaches the analytical solution.  If such a simulation focuses on
the density core expansion and any small scale structures, such as
cometary structures and photo-dissociation regions, one can cautiously
sacrifice temporal accuracy at large scales for computational savings.

\subsubsection{Change of incident radiation}
\label{sec:dt_tau}

Ionization front velocities can approach significant fractions of the
speed of light in steep density gradients and in the early expansion
of the \ion{H}{2} region.  If the ionization front position is
critical to the calculation, the radiation transport timestep can be
derived from an estimate of the ionization front velocity
%
\begin{equation}
  \label{eqn:dt_incident}
  v_{\rm IF}(\mathbf{r}) \approx \frac{F(\mathbf{r})}
  {n_{\rm abs}(\mathbf{r})},
\end{equation}
%
based on the incident radiation field at a particular position.  The
timestep is then chosen so that the ionization front only crosses one
cell per timestep $dt_{\rm P, cell} = dx / v_{\rm IF}$, and the global
timestep is $dt_{\rm P} = \min(dt_{\rm P, cell})$.  

To make this approach computationally straightforward, we consider a
spherically symmetric case without recombinations.  Thus all photons
emitted from the source result in an ionization (also see
\S\ref{sec:dt_hi}).  Ultimately in any approach, we are focused on
accurately following the time evolution of the radiation field.  In
this method, we base the timestep directly on the change in radiation
field, the specific intensity, $I/\dot{I}$.  We consider the field
after propagating through the cell, so $I = I_0 \exp(-\tau)$, where
$\tau = n_{\rm H} \sigma dl$ is the optical depth through the cell.
The change in specific intensity is
%
\begin{equation}
  \label{eqn:vifront2}
    \frac{dI}{dt} = I_0 \exp(-\tau) (-\dot{n}_{\rm H} \sigma dl),
\end{equation}
%
which can be expressed in terms of local optical depth and neutral
fraction,
\begin{equation}
  \label{eqn:vifront3}
    \frac{dI}{dt} = -I \frac{\tau \dot{n}_{\rm H}}{n_{\rm H}}.
\end{equation}
%
This results in a local timestep
%
\begin{equation}
  \label{eqn:vifront4}
  dt_P = \frac{I}{|dI/dt|} = C_{\rm RT,cfl} \frac{n_{\rm H}}{\tau \dot{n}_{\rm H}},
\end{equation}
%
where $C_{\rm RT,cfl}$ is a safety factor that restricts the change in
intensity to its inverse.  In practice, we have found that a ceiling
of 3 can be placed on the optical depth, so optically thick cells do
not create a very small timestep.  We still find excellent agreement
with analytical solutions with this approximation.  We show the
accuracy using this timestep method in \S\ref{sec:dt_dep}.

\subsection{Parallelization Strategy}

Parallelization of the ray tracing code is essential when exploring
problems that require high resolution and thus large memory
requirements.  Furthermore, \enzo~is already parallelized and scalable
to $O(10^2)$ processors in AMR simulations, and $O(10^3)$ in unigrid
calculations.  \enzo~stores the AMR grid structure on every processor,
but only one processor contains the actual grid and particle data and
photon packages.  All other processors contain an empty grid
container.  As discussed in \textit{Step 4} in \S\ref{sec:design}, we
store the photon packages that need to be transferred to other grids
in the linked list \texttt{PhotonMoveList}.  In a single processor
(serial) run, moving the rays is trivial by inserting these photons
into the linked list of the destination grid.  For multi-processor
runs, we must send these photons through MPI communication to the
processors that host the data of the destination grids.  We describe
our strategy below.

The easiest case is when the destination grid exists on the same
processor as the source grid, where we move the ray as in the serial
case.  For all other rays, we organize the rays by destination
processors and send them in groups.  We also send the destination grid
level and ID number along with the ray information, which is listed at
the beginning of \S\ref{sec:design}.  

For maximum overlap of communication and computation, which enables
scaling to large numbers of processors, we must employ
``non-blocking'' MPI communication, where each processor does not wait
for synchronization with other processors.  We use this technique for
the sending and receiving of rays.  Here we desire to minimize the
idle time of each processor when it is waiting to receive data.  In
the loop shown in Figure \ref{fig:MainAlgorithm} with the conditional
that checks whether we have traced all of the rays, we aggressively
transport rays that are local on the processor, and process any MPI
receive calls as they arrive, not waiting for their completion in
order to continue to the next iteration.  We describe the steps in
this algorithm next.

\step{1} Before any communication occurs, we count the number of rays
that will be sent to each processor.  The MPI receive calls
(\texttt{MPI\_Irecv}) must have a data buffer that is greater than or
equal to the size of the message.  We choose to send a maximum of
$N_{\rm max}$ (= 10$^5$ in \enzo~\texttt{v2.0}) rays per MPI message.
Therefore, we allocate a buffer of this size for each
\texttt{MPI\_Irecv} call.  We then determine the number of MPI messages
$N_{\rm mesg}$ and send this number in a non-blocking fashion,
i.e. \texttt{MPI\_Isend}.

\step{2} Pack the photon packages into a contigious array for MPI
communication while the messages from Step 1 completes.

\step{3} Process the number of photon messages that we are expecting
from each processor, sent in Step 1.  Then post this number of
\texttt{MPI\_Irecv} calls for the photon data.  Because we strive to
make the ray tracing routine to be totally non-blocking, the
processors will most likely not be synchronized on the same loop
(Steps 3--5 in \S\ref{sec:design}).  Therefore, there might be
additional $N_{\rm mesg}$ MPI messages waiting to be processed.  We
check for these messages and aggressively drain the message stack to
determine the total number of photon messages that we are expecting
and post their associated \texttt{MPI\_Irecv} calls for the photon
data.

\step{4} Send the grouped photon data with \texttt{MPI\_Isend} with a
maximum size of $N_{\rm max}$ photons.

\step{5} Place any received photon data into the destination grids.
We monitor whether the processor has any rays that were moved to grids
on the same processor.  If so, this processor has rays to transport,
and we do not necessarily have to wait for any MPI receive messages
and thus use \texttt{MPI\_Testsome} to receive any messages that have
already arrived.  If not, we call \texttt{MPI\_Waitsome} to wait for
any MPI receive messages.

\step{6} If all processors have exhausted their workload, then all
rays have been either absorbed, exited the domain, or halted after
travelling a distance $cdt_P$.  We check this in a similar
non-blocking manner as the $N_{\rm mesg}$ calls in Step 1.

Lastly we have experimented with a hybrid OpenMP/MPI version of \enzo,
where workload is partitioned over grids on each MPI process.  We
found that parallelization over grids for the photon transport does
not scale well, and threading over the rays in each grid is a better
approach.  Because the rays are stored in a linked list in each grid,
we must manually split the list into separate lists and let each
thread work on each list.

\section{Radiative Transfer Tests}
\label{sec:rt_tests}

Tests plays an important role in creating and maintaining
computational tools.  In this section, we present tests drawn from the
Cosmological Radiative Transfer Codes Comparison Project
\citep[][hereafter RT06]{RT06}, where results from 11 different
radiative transfer codes compared results in four test problems.  The
codes use various methods for radiation transport: ray tracing with
short, long, and hybrid characteristics, Monte Carlo casting;
ionization front tracking \citep{Alvarez06_IFT}; variable Eddington
Tensor formalism \citep{Gnedin01_OTVET}.  They conducted tests that
investigated (1) the growth of a single Str\"{o}mgren sphere enforcing
isothermal conditions, (2) the same test with an evolving temperature
field, (3) shadowing created by a dense, opticlally thick clump, and
(4) multiple \ion{H}{2} regions in a cosmological density field.  In
all of the tests presented here, we use the method of restricted
neutral fraction changes (\S\ref{sec:dt_hi}) for choosing a radiative
transfer timestep.  We cast 48 rays (HEALPix level 1) from the point
source and require a sampling of at least $\Phi_c = 5.1$ rays per
cell.

\subsection{Test 1. Pure hydrogen isothermal \ion{H}{2} region
  expansion}
\label{sec:test1}

\begin{figure}[t]
  \plottwo{fig/test1_ionfrac.eps}{fig/test1_ifront.eps}
  \caption{\label{fig:test1_ifront} Test 1 (\ion{H}{2} region
    expansion with a monochromatic spectrum of 13.6 eV).  Left:
    Radially averaged profile of neutral (solid) and ionized (dashed)
    fraction at 10, 30, 100, and 500 Myr.  Right: Evolution of the
    calculated (top, dashed) and analytical (top, solid) Str\"{o}mgren
    radius.  The ratio of these radii are plotted in the bottom panel.}
\end{figure}

\begin{figure}[t]
  \epsscale{0.8}
  \plotone{fig/test1_HI500.eps}
  \caption{\label{fig:test1_HI} Test 1 (\ion{H}{2} region expansion
    with a monochromatic spectrum of 13.6 eV). Slice of neutral
    fraction at the origin at 500 Myr.}
  \epsscale{1.0}
\end{figure}

The expansion of an ionizing region with a central source in a uniform
medium is a classic problem first studied by \citet{Stroemgren39}.
This simple but useful test can uncover any asymmetries or artifacts
that may arise from deficiencies in the method or newly introduced
bugs in the development process.  In this problem, the ionized region
grows until recombinations balance photo-ionizations, which happens at
a radius of
%
\begin{equation}
  \label{eqn:Rstr}
  R_s = \left(\frac{3\dot{N}_\gamma}{4\pi \alpha_{\rm B} n_H^2}\right)^{1/3},
\end{equation}
where $\dot{N}_\gamma$ is the ionizing photon luminosity, $\alpha_{\rm
  B}$ is the recombination rate, and $n_H$ is the ambient hydrogen
number density.  The evolution of the radius $r_s$ and velocity $v_s$
of the ionization front has an exact solution of
%
\begin{equation}
  \label{eqn:rstr}
  r_s(t) = R_s [ 1 - \exp(-t/t_{\rm rec}) ]^{1/3}
\end{equation}
\begin{equation}
  \label{eqn:vstr}
  v_s(t) = \frac{R_s}{3t_{\rm rec}} \frac{\exp(-t/t_{\rm rec})} {[1 -
    \exp(-t/t_{\rm rec})]^{2/3}},
\end{equation}
where $t_{\rm rec} = (\alpha_{\rm B} n_H)^{-1}$ is the recombination
time.

We adopt the problem parameters used in RT06.  The ionizing
source emits $5 \times 10^{48}$ ph s$^{-1}$ of monochromatic radiation
at 13.6 eV and is located at the origin in a simulation box of 6.6
kpc.  The ambient medium is initially set at $T=10^4$ K, $n_H =
10^{-3} \cubecm$, $x = 1.2 \times 10^{-3}$, resulting in $R_s = 5.4$
kpc and $t_{\rm rec} = 122.4$ Myr.  The problem is run for 500 Myr.
In the original tests, the temperature is fixed at $10^4$ K; however,
our solver is inherently tied to the chemistry and energy solver.  To
mimic an isothermal behavior, we set the adiabatic index $\gamma =
1.0001$, which ensures an isothermal state but not a fixed ionization
fraction outside of the Str\"{o}mgren sphere.

In Figure \ref{fig:test1_ifront}, we show (a) the evolution of the
neutral and ionization fraction as a function of radius at $t = $ 10,
30, 100, and 500 Myr, and (b) the growth of the ionization front
radius as a function of time and its ratio with the analytical
Str\"{o}mgren radius (Eq. \ref{eqn:rstr}).  The ionization front has a
width of $\sim0.7$ kpc, which is in agreement with the inherent
thickness of $\sim18 \lambda_{\rm mfp} = 0.74$ kpc, given a 13.6 eV
mono-chromatic spectrum.  There are small kinks in the neutral
fraction at 1.5 and 3 kpc that corresponds to artifacts created by the
photon package splitting at these radii.  However these do not affect
the overall solution.  One difference between our results and the
codes presented in RT06 is the increasing neutral fraction outside of
the \ion{H}{2}.  This occurs because the initial ionized fraction and
temperature is set to $1.3 \times 10^{-3}$ and 8000 K, which are not
the equilibrium values.  Over the 500 Myr in the calculation, the
neutral fraction increases to 0.2, which is close to its equilibrium
value.  In the right panel of Figure \ref{fig:test1_ifront}, the
ionization front radius exceeds $R_s$ by a few percent for most of the
calcuation.  This difference happens because the analytical solution
(Eq. \ref{eqn:rstr}) assumes the \ion{H}{2} region has a constant
ionized fraction.  The evolution of the ionized fraction as a function
of radius can be analytically calculated \citep[e.g.][]{Osterbrock89,
  Petkova09}, causing the ionization front radius to be slightly
larger, increasing from 0 to 3\% in the interval 80--350 Myr.  Our
results are in excellent agreement with this more accurate analytical
solution.  In Figure \ref{fig:test1_HI}, we show a slice of the
neutral fraction through the origin.  Other than the ray splitting
artifacts that generate the plateaus at 1.5 and 3 kpc, one sees
spherical symmetry in our solution.

\subsection{Test 2. \ion{H}{2} region expansion: temperature evolution}

\begin{figure}[t]
  \plottwo{fig/test2_ionfrac.eps}{fig/test2_neutral.eps}
  \caption{\label{fig:test2_1} Test 2. (\ion{H}{2} region
    expansion with a $T=10^5$ K blackbody spectrum).  Left:
    Radially averaged profile of neutral (solid) and ionized (dashed)
    fraction at 10, 30, 100, and 500 Myr.  Right: Evolution of the
    average neutral fraction.}
\end{figure}

\begin{figure}[t]
  \plotone{fig/test2_ifront.eps}
  \caption{\label{fig:test2_2} Test 2. (\ion{H}{2} region expansion
    with a $T=10^5$ K blackbody spectrum).  Top: Evolution of the
    radius of the simulated ionization front (dashed) and analytical
    (solid) Str\"{o}mgren radius.  Bottom: The ratio of the calculated
    and analytical Str\"{o}mgren radius.}
\end{figure}

\begin{figure}[]
  \plotone{fig/test2_slices}
  \caption{\label{fig:test2_3} Test 2. (\ion{H}{2} region expansion
    with a $T=10^5$ K blackbody spectrum).  Top: Slices through the
    origin of neutral fraction at 10 and 100 Myr.  Bottom: Slices of
    temperature at 10 and 100 Myr.}
\end{figure}

This test is similar Test 1, but the temperature is allowed to evolve
The radiation source now has a blackbody spectrum with a $T = 10^5$ K.
The initial temperature is set at 100 K.  The higher energy photons
have a longer mean free path than the photons at the ionization
threshold in Test 1.  Thus the ionization front is thicker as the
photons can penerate deeper into the neutral medium.  Here we use 4
energy groups with the following mean energies and relative
luminosities: $E_i = (16.74, 24.65, 34.49, 52.06)$, $L_i/L = (0.277,
0.335, 0.2, 0.188)$.

In Figure \ref{fig:test2_1}, we show the neutral and ionized fraction
as a function of radius at $t = $ 10, 30, 100, and 500 Myr, and the
total neutral fraction of the domain.  Compared with Test 1, the
ionization front is thicker, as expected with the harder spectrum.
The total neutral fraction decreases to 0.67 over 4$t_{\rm rec} = 500$
Myr, which is in agreement with the analytical expectation and other
codes in RT06.  In Figure \ref{fig:test2_2}, we show the ratio of the
ionization front radius $r_{\rm IF}$ in our simulation and $R_s$.
Before $1.5t_{\rm rec}$, $r_{\rm IF}$ lags behind $R_s$, initially by
10\% and then increases to $R_s$; however afterwards, this ratio
asymptotes to a solution that is 4\% greater than $R_s$.  This
behavior puts our code is approximately the median result in RT06,
where this ratio varies between 1 and 1.1, and the early evolution of
$r_{\rm IF}$ is underpredicted by almost all of the codes.  If we use
one energy group with the mean energy (29.6 eV) of a $T=10^5$ K
blackbody, we find that $r_{\rm IF}/R_s$ = 1.08, which is
representative of the codes in the upper range of RT06.  In Figure
\ref{fig:test2_3}, we show slices of neutral fraction and temperature
through the origin at $t = $ 10 and 100 Myr.  Here one sees the
spherically symmetric \ion{H}{2} regions and a smooth temperature
transition to the neutral ambient medium.

\subsection{Test 3. I-front trapping in a dense clump and the
  formation of a shadow}
\label{sec:test3}

\begin{figure}[t]
  \plottwo{fig/test3_HI.eps}{fig/test3_temp.eps}
  \caption{\label{fig:test3_1} Test 3. (I-front trapping in a dense
    clump and shadowing).  Left panel: Line cut from the point source
    through the middle of the dense clump at $t = 1, 3, 5, 15$ Myr.
    of the average neutral fraction (left) and temperature
    (right) of the clump.}
\end{figure}

\begin{figure}[t]
  \plotone{fig/test3_clump.eps}
  \caption{\label{fig:test3_2} Test 3. (I-front trapping in a dense
    clump and shadowing).  Evolution of the average ionized fraction
    (top) and temperature (bottom) of the overdense clump.}
\end{figure}

\begin{figure}[t]
  \plotone{fig/test3_slices}
  \caption{\label{fig:test3_3} Test 3. (I-front trapping in a dense
    clump and shadowing).  Clockwise from upper left: Slices through
    the origin of neutral fraction (1 Myr), temperature (1 Myr),
    temperature (15 Myr), and neutral fraction (15 Myr).}
\end{figure}

The diffusivity and angular resolution of a radiative transport method
can be tested with the trapping of an ionization front by a dense,
neutral clump.  In this situation, the ionization front will uniformly
propagate until it reaches the clump surface.  Then the radiation in
the line of sight of the clump will be absorbed more than the ambient
medium.  If the clump is optically thick, a shadow will form behind
the clump.  The sharpness of the ionization front at the shadow
surface can be used to determine the diffusivity of the method.
Furthermore the shadow surface should be aligned with the outermost
neutral regions of the clump, which can visually assess the angular
resolution of the method.

The problem for this test is contained in a 6.6 kpc box with an
ambient medium of $n_H = 2 \times 10^{-4} \cubecm$ and $T = 8000$ K.
The clump is in pressure equilibrium with $n_H = 0.04 \cubecm$ and $T
= 40$ K.  It has a radius of $r_c = 0.8$ kpc and is centered at
$(x,y,z) = (5, 3.3, 3.3)$ kpc.  In RT06, the test considered a plane
parallel radiation field with a flux $F_0 = 10^6$ ph s$^{-1}$
cm$^{-2}$ originating from the $y=0$ plane.  Our code can only
consider point sources, so we use a single radiation source located in
the center of the $y=0$ boundary.  The luminosity of $\dot{N}_\gamma =
3 \times 10^{51}$ ph s$^{-1}$ corresponds to the same flux $F_0$.  The
location where the ionization front trapping can be calculated
analytically \citep{Shapiro04}, and with these parameters, it should
halt at approximately the center of the clump.  We use the same four
energy groups as in Test 2.

In Figure \ref{fig:test3_1}, we show neutral fraction and temperature
of a one-dimensional cut through the center of the dense clump at $z =
0.5$ at $t = 1, 3, 5, 15$ Myr.  The ionization front is halted at a
little more than halfway through the clump, which is consistent with
the analytical expectation.  The hardness of the $T = 10^5$ K
blackbody spectrum allows the gas outside of the ionization front.
Where the gas is ionized, the temperature is between 10,000 and 20,000
K, but decreases sharply with the ionized fraction.  Figure
\ref{fig:test3_2} depicts the average ionized fraction and temperature
inside the dense clump, which both gradually increase as the
ionization front propagates through the overdensity.  Our results are
consistent with RT06.  Finally we show slices of neutral fraction and
temperature in the $z = 0.5$ plane in Figure \ref{fig:test3_3}.  The
neutral fraction slices prominently show the sharp shadows created by
the clump and demonstrates the non-diffusivity behavior of ray
tracing.  The discretization of the sphere creates one neutral cell on
the left side of the sphere.  This inherent artifact to the initial
setup carries through the calculation.  We did not smooth the clump
surface like in some of the RT06 codes, in order to remove this
artifact.  It is seen in the neutral fraction and temperature states
at all times and is not a caused by our ray tracing algorithm.

\subsection{Test 4. Multiple sources in a cosmological density field}

\begin{figure}[t]
  \epsscale{0.75}
  \plotone{fig/test4_ion}
  \caption{\label{fig:test4_1} Test 4. (Multiple cosmological
    sources).  Evolution of the mass- (dashed) and volume-averaged
    (solid) ionized fraction.}
  \epsscale{1}
\end{figure}

\begin{figure}[t]
  \epsscale{0.9}
  \plotone{fig/test4_slices}
  \caption{\label{fig:test4_2} Test 4. (Multiple cosmological
    sources).  Top: Slices through the origin of neutral fraction at
    50 and 200 kyr at the coordinate $z = z_{\rm box}/2$.  Bottom:
    Slices of temperature at 50 and 200 kyr.  No smoothing has been
    applied to the images.}
  \epsscale{1}
\end{figure}

The last test in RT06 involves a static cosmological density field at
$z=9$.  The simulation comoving box size is 0.5 $h^{-1}$ Mpc and has a
resolution of 128$^3$.  There are 16 point sources that are centered
in the 16 most massive halos.  They emit $f_\gamma = 250$ ionizing
photons per baryon in a blackbody spectrum with an effective
temperature $T = 10^5$ K, and they live for $t_s = 3$ Myr.  Thus the
luminosity of each source is
%
\begin{equation}
  \label{eqn:cosmo_lum}
  \dot{N}_\gamma = f_\gamma \frac{M \Omega_b} {\Omega_m m_H t_s},
\end{equation}
where $M$ is the halo mass, $\Omega_m = 0.27$, and $\Omega_b =
0.043$.  The radiation boundaries are isolated so that the radiation
leaves the box instead being shifted periodically.  The simulation is
evolved for 0.4 Myr.

We show the growth of the \ion{H}{2} regions by computing the
mass-averaged $x_m$ and volume-averaged $x_v$ ionized fraction in
Figure \ref{fig:test4_1}.  Initially $x_m$ is larger than $x_v$, and
at $t \sim 170$ kyr, the $x_v$ becomes larger.  This is indicative of
inside-out reionization \citep[e.g.][]{Gnedin00, Miralda00,
  Sokasian03}, where the dense regions around halos are ionized first,
then the voids are ionized last.  At the end of the simulation, $x_v =
0.65$, which is lower than all of the codes in RT06.  However by
visual inspection in the slices of electron fraction
(Fig. \ref{fig:test4_2}), there appears to be very good agreement with
C$^2$-ray and FTTE.  By first glance, our result appears to be
different than the RT06 because of the color-mapping.  Our results are
also in good agreement with the multi-frequency version of TRAPHIC
\citep[][see also for better representations of the electron fraction
  slices]{Pawlik10}.  In the slices of electron fraction and
temperature, Figure \ref{fig:test4_2}, the photo-heated regions are
larger than the ionized regions by a factor of 2--3 because of the
hardness of the $T = 10^5$ K blackbody spectrum.

\section{Radiation Hydrodynamics Tests}
\label{sec:radhydro}

In this section, we show results from radiation hydrodynamics test
problems presented in \citet[][hereafter RT09]{Iliev09}.  They involve
(1) the expansion of an \ion{H}{2} region in a uniform medium, similar
to Test 2, (2) an \ion{H}{2} region in an isothermal sphere, and (3)
the photo-evaporation of a dense, cold clump, similar to Test 3.  We
expand on this test suite to include more complex situations, such as
a Rayleigh-Taylor problem illuminated by a radiation source, champange
flows, collimated radiation, an irradidated blast wave, and an
\ion{H}{2} region in a rotating sphere.  For the \citeauthor{Iliev09}
tests, we turn off self-gravity and AMR in accordance with them.  In
the latter tests, we will indicate whether we use those capabilities
of \enzo.  Lastly, we use the new MHD implementation in \enzo~v2.0 in
the problem of a growing \ion{H}{2} region in a magnetic field.

\subsection{Test 5: Classical \ion{H}{2} region expansion}
\label{sec:test5}

\begin{figure}[t]
  \epsscale{0.75}
  \plotone{fig/test5_ifront}
  \caption{\label{fig:test5_1} Test 5. (\ion{H}{2} region in a uniform
    medium).  Top: Growth of the computed ionization front radius at
    an ionized fraction $x_e = 0.5$ (dashed) and at a temperature $T =
    10^4$ K (dotted) compared to the analytical estimate (solid;
    Eq. \ref{eqn:test5_r}).  Bottom: The ratio of the computed
    ionization front radii to the analytical estimate.} 
  \epsscale{1}
\end{figure}

\begin{figure}[t]
  \plotone{fig/test5_profiles}
  \caption{\label{fig:test5_2} Test 5. (\ion{H}{2} region in a uniform
    medium).  Clockwise from the upper left: Radial profiles of
    density, temperature, ionized fraction, and pressure at times $t
    =$ 10, 200, and 500 Myr.}
\end{figure}

\begin{figure}[t]
  \plotone{fig/test5_slices}
  \caption{\label{fig:test5_3} Test 5. (\ion{H}{2} region in a uniform
    medium).  Clockwise from the upper left: Slices through the origin
    of ionized fraction, neutral fraction, temperature, and density at
    time $t =$ 500 Myr.}
\end{figure}

Here we consider the expansion of an \ion{H}{2} region into a uniform
neutral medium including the hydrodynamical response to the heated
gas.  The ionized region has a greater pressure than the ambient
medium, causing it to expand.  This is a well-studied problem
\citep{Spitzer78} with an analytical solution, where the ionization
front moves as
%
\begin{equation}
  \label{eqn:test5_r}
  r_s(t) = r_{s,0} \left(1 + \frac{7c_s}{4R_s}\right)^{4/7},
\end{equation}
where $c_s$ is the sound speed of the ionized gas and $r_{s,0}$ is the
$r_s$ in Equation \ref{eqn:rstr}.  The bubble eventually reaches
pressure equilibrium with the ambient medium at a radius
%
\begin{equation}
  \label{eqn:test5_final}
  r_f = R_s \left(\frac{2T}{T_0}\right)^{2/3},
\end{equation}
where $T$ and $T_0$ are the ionized and ambient temperatures,
respectively.  These solutions only describe the evolution at late
times, and not the fast transition from R-type to D-type at early
times.

The simulation setup is similar to Test 2 with the exception of the
domain size $L = 15$ kpc.  Here pressure equilbrium occurs at $r_f =
185$ kpc, which is not captured by this test.  However more
interestingly, the transition from R-type to D-type is captured and
occurs around $R_s = 5.4$ kpc.  The test is run for 500 Myr.

The growth of the \ion{H}{2} region is shown in Figure
\ref{fig:test5_1}, using both $T = 10^4$ K and $x_e = 0.5$ as
ionization front definitions, compared to the analytical solution
(Eq. \ref{eqn:test5_final}).  We define this alternative measure
because the ionization front becomes broad as the D-type front creates
a shock.  Densities in this shock, as seen in Figure
\ref{fig:test5_2}, are high enough for the gas to recombine but not
radiatively cool.  Before $2t_{\rm rec} \approx 250$ Myr, the
temperature cutoff overestimates $r_s$ by over 10\%; however at later
times, it provides a good match to the $t^{4/7}$ growth at late times.
With the $x_e = 0.5$ criteron for the ionization front, the radius is
always underestimated by $\sim20\%$.  This behavior was also seen in
RT09.

Figure \ref{fig:test5_2} shows the progression of the ionization front
at times $t$ = 10, 200, and 500 Myr in radial profiles of density,
temperature, pressure, and ionized fraction.  The initial \ion{H}{2}
region is overpressurized and creates an outward shock wave.  The
high-energy photons can penerate through the shock and partially
ionizes and heats the exterior gas, clearly seen in the profiles.  As
noted in RT09, this heated exterior gas creates an photo-evaporative
flow that flows inward.  This interacts with the primary shock and
creates the double-peaked features in the density profiles at 200 and
500 Myr.  Figure \ref{fig:test5_3} shows slices through the origin of
the same quantities, including neutral fraction.  These depict the
very good spherical symmetry of our method.  The only apparent
artifact is a very slight diagonal line, which is caused by the
HEALPix pixelization differences between the polar and equatorial
regions.  This artifact diminishes as the ray-to-cell sampling is
increased.

\subsection{Test 6: \ion{H}{2} region expansion in an isothermal
  sphere}

\begin{figure}[t]
  \epsscale{0.75}
  \plotone{fig/test6_ifront}
  \caption{\label{fig:test6_1} Test 6. (\ion{H}{2} region in a 1/$r^2$
    density profile).  Top: Growth of the computed ionization front
    radius as $T=10^4$ K and $x_e = 0.5$ as definitions for the front.
    Bottom: Velocity of the ionization front, computed from outputs at
    0.5 Myr intervals.  The velocity is calculated from $r_{\rm IF}$,
    whose coarse time resolution causes the noise seen in $v_{\rm
      IF}$.  It is smooth within the calculation itself.}
  \epsscale{1}
\end{figure}

\begin{figure}[t]
  \plotone{fig/test6_profiles}
  \caption{\label{fig:test6_2} Test 6. (\ion{H}{2} region in a 1/$r^2$
    density profile).  Clockwise from the upper left: Radial profiles of
    density, temperature, ionized fraction, and pressure at times $t
    =$ 3, 10, and 25 Myr.}
\end{figure}

\begin{figure}[t]
  \plotone{fig/test6_slices}
  \caption{\label{fig:test6_3} Test 6. (\ion{H}{2} region in a 1/$r^2$
    density profile).  Clockwise from the upper left: Slices through the origin
    of ionized fraction, neutral fraction, temperature, and density at
    time $t =$ 25 Myr.}
\end{figure}

A more physically motivated scenario is an isothermal sphere with a
constant density $n_c$ core, which is applicable to collapsing
molecular clouds and cosmological halos.  The radial density profile
is described by
%
\begin{equation}
  \label{eqn:test6_rho}
  n(r) = \left\{ \begin{array}{l@{\quad}l}
      n_c & (r \le r_0)\\
      n_c (r/r_0)^{-2} & (r > r_0)
    \end{array} \right. ,
\end{equation}
where $r_0$ is the radius of the core.  If the Str\"{o}mgren radius is
smaller than the core radius, then the resulting \ion{H}{2} region
never escapes into the steep density slope.  When the ionization front
propagates out of the core, it accelerates as it travels down the
density gradient.  There exists no analytical solution for this
problem with full gas dynamics but was extensively studied by
\citet{Franco90}.  After the gas is ionized and photo-heated, the
density gradient provides the pressure imbalance to drive the gas
outwards.

This test is constructed to study the transition from R-type to D-type
in the core and back to R-type in the density gradient.  Thus the
simulation focuses on small-scale, not long term, behavior of the
ionization front.  The simulation box has a side length $L = 0.8$ kpc
with core density $n_0 = 3.2\;\cubecm$, core radius $r_0 = 91.5$ pc
(15 cells), and temperature $T = 100$ K throughout the box.  The
ionization fraction is initially zero, and the point source is located
at the origin with a luminosity of $10^{50}$ ph s$^{-1}$ \cubecm.  The
simulation is run for 75 Myr.

Because this problem does not have an analytical solution, we compare
our calculated ionization front radius and velocity, shown in Figure
\ref{fig:test6_1}, to the RT09 results.  Their evolution are in
agreement within 5\% of RT09.  As in Test 5, we use an extra
definition of $T=10^4$ K for the ionization front.  We compute the
ionization front velocity from the radii at 50 outputs, which causes
the noise seen Figure \ref{fig:test6_1}.

For the first Myr, the radiation source creates a weak R-type front
where the medium is heated and ionized but does not expand because
$v_{\rm IF} > c_s$.  When $v_{\rm IF}$ becomes subsonic, the medium
can react to the passing ionization front and creates a shock, leaving
behind a heated an rarefied medium.  This behavior is clearly seen in
the radial profiles of density, temperature, ionized fraction, and
pressure in Figure \ref{fig:test6_2}.  The inner density decreases
over two order of magnitude after 25 Myr.  To illustrate any
deviations in spherical symmetry, we show in Figure \ref{fig:test6_3}
slices of density, temperature, neutral fraction, and ionized fraction
at the final time.  The only artifact apparent to us is the slight
broadening of the shock near the $x=0$ and $y=0$ planes.  This causes
the ionization front radius to be slightly smaller in those
directions.  In the diagonal direction, the neutral column density
through the shock is slightly smaller, allowing the high-energy
photons to photoionize and photoheat the gas to $x_e = 5 \times
10^{-3}$ and $T = 2000$ K out to $\sim50$ pc from the shock.  The
reflecting boundaries in Enzo might be responsible for this artifact
because this is not seen when the problem is centered in the domain,
removing any boundary effects.

\subsection{Test 7: Photo-evaporation of a dense clump}

\begin{figure}[t]
  \plotone{fig/test7_lines}
  \caption{\label{fig:test7_1} Test 7. (Photo-evaporation of a dense
    clump).  Line cuts from the point source through the middle of the
    dense clump at $t = 1, 10, 50$ Myr of (clockwise from the upper
    left) density, temperature, pressure, and neutral fraction.}
\end{figure}

\begin{figure}[t]
  \plotone{fig/test7_slices10}
  \caption{\label{fig:test7_2} Test 7. (Photo-evaporation of a dense
    clump).  Clockwise from the upper left: Slices through the clump
    center of neutral fraction, pressure, temperature, and density at
    time $t =$ 10 Myr.}
\end{figure}

\begin{figure}[t]
  \plotone{fig/test7_slices50}
  \caption{\label{fig:test7_3} Test 7. (Photo-evaporation of a dense
    clump).  Same as Figure \ref{fig:test7_1} but at $t = 50$ Myr.}
\end{figure}

The photo-evaporation of a dense clump in a uniform medium proceeds
very differently when radiation hydrodynamics is considered instead of
a static density field.  The ionization front first proceeds as a very
fast R-type front, then it slows to a D-type front when it encounts
the dense clump.  As the clump is gradually photoionized and heated,
it expands into the ambient medium.  The test presented here is
exactly like Test 3 but with gas dynamics.  In this setup, the
ionization front overtakes the entire clump, which is then completely
photo-evaporated.

Figure \ref{fig:test7_1} shows cuts of density, temperature, neutral
fraction, and pressure in a line connecting the source and the clump
center at $t = 1$, 10, and 50 Myr.  At 1 Myr, the ionization front has
propagated through the left-most 500 pc of the clump.  This heated gas
is now overpressurized, as seen in the pressure plot in Figure
\ref{fig:test7_1}, and then expands into the ambient medium.  This
expansion creates a photo-evaporative flow, seen in many star forming
regions \citep[e.g. the Carina Nebula][]{Ref?} as stars irradiate
nearby cold, dense overdensities.  These flows become evident in the
density at 10 Myr, seen both in the line cuts and slices (Figure
\ref{fig:test7_2}).  These outflows have temperatures up to 50,000 K.
At this time, the front has progressed about halfway through the
clump, if one inspects the neutral fraction.  However the high energy
photons have heated all but the rear surface of the clump.  At the end
of the test ($t = 50$ Myr), only the core and its associated shadow is
neutral, as seen in Figure \ref{fig:test7_3}.  The core has been
compressed by the surrounding warm medium, thus causing the higher
densities seen at $t = 50$ Myr.  The non-spherical artifacts on the
inner boundary of the warm outermost shell are caused by the initial
discretization of the sphere, as discussed in \S\ref{sec:test3}.

\subsection{Test 8: Champange flow from a dense clump}

\begin{figure}[t]
  \plotone{fig/test8_slices}
  \caption{\label{fig:test8_1} Test 8. (Champange flow from a dense
    clump).  Slices of density through the initial clump center in the
    x-y plane (top) and x-z plane (bottom) at $t = 10, 40, 100, 150$
    kyr.  Notice the instabilities that grow from perturbations
    created while the \ion{H}{2} region is contained in the dense
    clump.}
\end{figure}

Radiation-driven outflows from overdensities, known as champange
flows, is a long studied problem \citep[e.g.][\S3.3]{Yorke86}.  We set
up a spherical tophat with an overdensity of 10 and radius of 1 pc in
a simulation box of 8 pc.  The ambient medium is 290 \cubecm~and 100
K.  The radiation source is offset from the overdensity center by 0.4
pc.  It has a luminosity of \tento{49} ph s$^{-1}$ and a $T=10^5$ K
blackbody spectrum.  The resulting Str\"{o}mgren radius is 0.33 pc,
just inside of the overdense clump.  These parameters are the same
used in \citet{Bisbas09}.  The entire domain initially has an ionized
fraction of \tento{-6}.  We do not consider self-gravity.  The
simulation has a resolution of 128$^3$ on base grid, and we refine the
grid up to 4 times if a cell has an overdensity of $1.5 \times 2^l$,
where $l$ is the AMR level.  The simulation is run for 150 kyr.

We show slices in the x-y and x-z planes of density in Figure
\ref{fig:test8_1} at $t = 10, 40, 100, 150$ kyr.  In the direction of
the clump center, the ionization front transitions from spherical to
parabolic after it escapes from the clump in the opposite direction.
At $t = 10$ kyr, the surface of the \ion{H}{2} region is just
contained within the overdensity.  In the x-z plane, there are density
perturbuations only above a latitude of 45 degrees.  We believe that
these are caused by the mismatch between HEALPix pixels and the
Cartesian grid, even with our geometric correction.  After the
ionization front escapes from the clump in the negative x-direction,
these perturbations grow from Rayleigh-Taylor instabilities as the gas
is accelerated when it exits the clump.  As the shock propagates
through the ambient medium, it is no longer accelerated and has a
nearly constant velocity, as seen in Test 6.  Thus these perturbations
are not as vunerable to Rayleigh-Taylor instabilities at this point.
The ambient medium and shock are always optically thick, even in the
directions of the bubbles.  \citeauthor{Bisbas09} found that the shock
fragmented and formed globules; however we find the density shell is
stable against such fragmentation.  To investigate this scenario
further, our next tests involve radiation driven Rayleigh-Taylor
instabilities.

\subsection{Test 9: Irradiated Rayleigh-Taylor instability}

\begin{figure}[t]
  \epsscale{0.5}
  \plotone{fig/test9thick_rho}
  \plotone{fig/test9thick_temp}
  \plotone{fig/test9thick_efrac}
  \caption{\label{fig:test9_thick} Test 9. (Irradiated Rayleigh-Taylor
    instability; optically thick case).  Slices at $y=0$ of density
    (top), temperature (middle), and electron fraction (bottom).  The
    source turns on at $t=0$.}
  \epsscale{1}
\end{figure}

\begin{figure}[t]
  \epsscale{0.5}
  \plotone{fig/test9thin_rho}
  \plotone{fig/test9thin_temp}
  \plotone{fig/test9thin_efrac}
  \caption{\label{fig:test9_thin} Same as Figure \ref{fig:test9_thick}
    but for the optically thin case.}
  \epsscale{1}
\end{figure}

Here we combine the classic case of a Rayleigh-Taylor instability and
an expanding \ion{H}{2} region.  The Rayleigh-Taylor instability
occurs when a dense fluid is being supported by a lighter fluid,
initially in hydrostatic equilibrium, in the presence of a constant
acceleration field.  This classic test alone evaluates how subsonic
perturbations evolve.  We consider the case of a single-mode
perturbation.  We let the system evolve without any radiation until
the perturbation grows considerably and then turn on the radiation
source.  These tests demonstrate that Enzo+Moray can follow a highly
dynamic system and resolve fine density structures.

We run two cases --- an optically-thick and optically-thin case.  In
the former, we take the parameter choices from past literature
\citep[e.g.][]{Liska03, Stone08} by setting the top and bottom halves
of the domain to a density $\rho_1 = 2$ and $\rho_0 = 1$,
respectively.  The velocity perturbation is set in the $z$-direction
by
\begin{equation}
  v_z(x,y,z) = 0.01 [1 + \cos(2\pi x / L_x)] \times 
  [1 + \cos(2\pi y / L_y)] \times [1 + \cos(2\pi z / L_z)]/8.
\end{equation}

We set the accleration field $g_z = 0.1$ and the adiabatic index
$\gamma = 1.4$.  We use a domain size of $(L_x, L_y, L_z) = (0.5, 0.5,
1.5)$ with a resolution of (64, 64, 192).  For hydrostatic equilbrium,
we set $P = P_0 - g \rho(z) z$ with $P_0 = 2.5$.  In order to consider
a radiation source with a ionizing photon luminosity of $10^{42}$ ph
s$^{-1}$, we scale the domain to a physical size of (0.5, 0.5, 1.5)
pc; time is in units of Myr; density is in units of $m_h$, resulting
in an initial temperature of $(T_0, T_1) = (363, 726)$ K.  The
radiation source starts to shine at $t = 10$ Myr.

The optically-thin case is set up similarly but with three
changes---(1) a density constrast of 10, (2) a luminosity of $10^{43}$
ph s$^{-1}$, and (3) the source is born at 6.5 Myr.  The time units
are decreased to 200 kyr so that $(T_0, T_1) = (1.8 \times 10^3, 1.8
\times 10^4)$ K.  Note that in code units, pressure is unchanged.  We
adjust the physical unit scaling, so the light fluid has $T > 10^4$ K,
$x_e \sim 1$, and thus optically-thin.  Furthermore, the ionization
front remains R-type before interacting with the instability.  A
possible physical analogue could be a radiation source heating and
rarefying the medium below.

The $x$ and $y$-boundaries are periodic, and the $z$-boundaries are
reflecting.  The radiation source is placed at the center of lower
$z$-boundary face.  The periodic boundaries will cause features that
are non-physical, considering a more realistic plane-parallel case.
Nevertheless, these tests provide a good check on a radiation
hydrodynamics solver.  We show the evolution of the density,
temperature, and ionized fraction of the optically thick and optically
thin cases in Figures \ref{fig:test9_thick} and \ref{fig:test9_thin}.
The initial state of the Rayleigh-Taylor instability is shown in the
left panels.

In the optically thick case, a D-type front is created, which is
clearly illustrated by the spherical density enhancement at 0.02 Myr.
The shock then passes through the instability at $\sim$0.25 Myr and
reflects off the upper $z$-boundary.  This and complex shock
reflections create a Richtmeyer-Meshkov instability, driving a chaotic
jet-like structure downwards.  The radiation source photo-evaporates
the outer parts of this structure.  The interaction between the dense
cool ``jet'' and the hot medium further drives instabilities along the
surface, which can be seen when comparing $t = 0.59$ Myr and $t =
0.91$ Myr slices.  At the latter time, the jet cannot reach the bottom
of the domain before being photo-evaporated.  Eventually this
structure is completely destroyed, leaving behind a turbulent medium
between the hot and cold regions.

The optically thin problem is less violent than the optically thick
case because the R-type front does not interact with the initial
instability as strongly.  The radiation source provides further
buoyance in the already $T=10^4$ K gas.  The gas first to be ionized
and photo-evaporates is the outer regions of the instability.  The
enhanced heating also drives the upper regions of the instability,
making the top interface turbulent.  It then reflects off the upper
$z$-boundary and creates a warm $T = 5 \times 10^3$ K, partially
ionized ($x_e \sim 10^{-2}$), turbulent medium, seen in the slices $t
\ge 0.67$ Myr.  The slices of electron fraction also show that the
dense gas is optically thick.

\subsection{Test 10: Photo-evaporation of a blastwave}

A supernova blastwave being irradiated by a nearby star is a likely
occurence in massive-star forming regions.  In this test, we set up an
idealized test that mimics this scenario.  The problem is similar to
Test 5 with two changes --- (1) the domain has been decreased to
$L_{\rm box} = 1.5$ kpc; and (2) a mild blastwave is initialized in
the center of the domain.  We use 3 levels of AMR with a base grid of
64$^3$, which is refined if the density slope is greater than 0.4.
The blastwave is initially in the free expansion phase with a total
energy of $10^{50}$ erg and total mass of $100 \Ms$, corresponding to
$E = 315$ eV per particle or $E/k_b = 3.66 \times 10^6$ K.  The
blastwave is setup with a linearly increasing velocity profile
according to \citet{Truelove99} with $v_{\rm max} = 130$ km s$^{-1}$
and a radius of $L_{\rm box}/50 = 30$ pc.  The blastwave initially has
a uniform density of $\rho_0 = 0.06 \cubecm$.  The initial kinetic
energy is converted into thermal energy as the shock propagates away,
and the blastwave transitions into the Sedov-Taylor phase at $r \sim
100$ pc and $t \sim 225$ kyr after it has accumulated an equal amount
of ambient material into the shock.  The simulation is run for 25 Myr.

\subsection{Test 11: Collimated radiation from a dense clump}

\begin{figure}[t]
  \epsscale{1}
  \plotone{fig/test11_slices}
  \caption{\label{fig:test11} Test 11 (Collimated radiation from a
    dense clump).  Slices of density (top) and temperature (bottom) at
    $t = 0.1, 3.25, 10.75, 23.25$ Myr.  The conical \ion{H}{2} region
    drives shocks transversely into the overdense sphere and creates
    polar champange flows.  The ambient medium is heated to $T \sim 3
    \times 10^4$ K as the ionization front passes the
    constant-pressure cloud surface.  The ionization front changes
    from D-type to R-type after it enters the ambient medium.}
  \epsscale{1}
\end{figure}

Some astrophysical systems produce collimated radiation either
intrinisically by relavistic beaming or by an optically-thick torus
absorbing radiation in the equatorial plane.  The latter case would be
applicable in a subgrid model of an AGN or a protostar, for example.
Simulating collimated radiation with ray tracing is trivially
accomplished by only initializing rays that are within some opening
angle $\theta_c$.

We use a domain that is 2 kpc wide and has an ambient medium with
$\rho_0 = 10^{-3}$ \cubecm, $T = 10^4$ K, $x_e = 0.99$.  We place a
dense clump with $\rho/\rho_0 = 100$, $T = 100$ K, $x_e = 10^{-3}$,
and $r = 250$ pc, at the center of the box.  Radiation is emitted in
two polar cones with $\theta_c = \pi/6$ with a total luminosity of
$10^{49}$ erg s$^{-1}$ and a 17.6 eV mono-chromatic spectrum.  This
results in $t_{\rm rec} = 1.22$ Myr and $R_s = 315$ pc, just outside
of the sphere.  The base grid has a resolution of 64$^3$, and it is
refined with the same overdensity criterion as Test 8.  We run this
test for 25 Myr.

We illustrate the expansion of the \ion{H}{2} region created by the
beamed radiation in Figure \ref{fig:test11}.  Before 3 Myr, the
\ion{H}{2} region is conical and contained within the dense clump,
depicted in the $t = 0.1$ Myr snapshot of the system.  At this time,
the ionization front is transitioning from R-type to D-type in the
transverse direction of the cone.  This can be seen in the minute
overdensities on the \ion{H}{2} transverse surface.  When it breaks
out of the overdensity, it creates a champange flow, where the
ionization front transitions back to a weak R-type front.  The cloud
surface is a constant-pressure contact discontinuity (CD) with a
density jump of 100.  After the front heats the gas at the CD, there
exists a pressure difference of $\sim 100$.  In response, the high
density gas accelerates into the ambient medium and heats it to $3
\times 10^4$ K.  Additionally a rarefaction wave travels towards the
clump center.  At later times, the transverse D-type front continues
through the clump, eventually creating a disk-like structure at the
final time.  The polar champange flows proceed to flow outwards,
creating a density shell with a diffuse ($10^{-28}$ \cubecm) and warm
(5000 K) medium in its wake.

% \subsection{Test 12: \ion{H}{2} region in a rotating medium}
% 
% Angular momentum is prevalent throughout all astrophysical systems.
% We construct a problem with an \ion{H}{2} region that is completely
% contained within a sphere with solid-body rotation.  Self-gravity is
% considered in this test.  The sphere initially has $R = 40$ pc with a
% uniform density $\rho = 200~\cubecm$, temperature $T = 50$ K, and
% electron fraction $x_e = 10^{-6}$.  The sphere is partially
% rotationally supported by a factor of 0.25, i.e. the tangential
% velocity $v_\theta(r) = r\sqrt{GM_{\rm tot}/R}/4$, where $M_{\rm tot}
% = 1.3 \times 10^6 \Ms$.  The sphere and radiation source are located
% in the center of the domain.  The sphere is in pressure equilibrium
% with the ambient medium that has $\rho_0 = 1 \cubecm$, $T = 10^4$ K,
% and $x_e = 0.99$.  The radiation source has an ionizing luminosity $L
% = 10^{50}$ ph s$^{-1}$ with a 17.6 eV mono-chromatic spectrum.  The
% test uses AMR with a base grid of 64$^3$ and 2 levels of refinement,
% using the same density criterion as in Test 8.

% \subsection{Test 12: \ion{H}{2} region in a static potential}

% Next we construct a problem that is similar to the Rayleigh-Taylor
% test problem (Test 9) in nature but is spherically symmetric.  We use
% the same physical setup as Test 5 with a uniform medium and the
% radiation source at the origin.  It has a luminosity of $5 \times
% 10^{48}$ ph s$^{-1}$ and a mono-chromatic spectrum of 14.6 eV.  We
% choose this energy so the ionization front is relatively sharp but
% still heats the gas to $10^4$ K.  The ambient medium has a density of
% $10^{-3}~\cubecm$ and an initial temperature of 100 K.  A point source
% of gravity also resides at the origin.  To have an equal gravitational
% acceleration $g(r) = g_0 / r^3$ at the Str\"{o}mgren radius $R_s =
% 5.4$ kpc ($r = 0.36$) as Test 9, we set the strength of the point
% source $g_0 = 4.67 \times 10^{-3}$.  The gravity point source has the
% same effect on any pertubations as the constant gravitational force in
% Rayleigh-Taylor instability test.

\subsection{Test 12: Time variations of the source luminosity}

Our implementation retains the time derivative of the radiative
transfer equation (Eq. \ref{eqn:rteqn}) if we choose a constant ray
tracing timestep, which saves the photon packages between timesteps if
the $c\;dt_{\rm P} < L_{\rm box}$.  This effect only becomes apparent
when the variation timescale of the point source is smaller than the
light crossing time of the simulation.  Furthermore, the timestep
should resolve the variation timescale by a few times.  To test this,
we can use an exponentially varying source with some duty cycle.  In a
functional form, this can be described as
%
\begin{equation}
  \label{eqn:test13}
  L(t) = L_{\rm max} \times \exp(t_f/t_0 - 1),
\end{equation}
%
where $t_f = \mathrm{mod}(t,t_0)$ and $t_0$ is the duty cycle.  To
illustrate the variations in the photo-ionization rates due to a
varying source, we remove any dependence on the medium by considering
an optically-thin uniform density.  We take $L_{\rm box} = 1$ Mpc,
which has a light crossing time of 3.3 Myr.  Thus we consider a source
at the origin with $L_{\rm max} = 10^{50}$ ph s$^{-1}$ and $t_0 = 0.5$
Myr.  We use a radiative transfer timestep of 50 kyr to resolve the
duty cycle by 10 timesteps.

\subsection{Test 14: \ion{H}{2} region with MHD}

\begin{figure}[t]
  \plotone{fig/test14_slices_xy}
  \caption{\label{fig:test14_1} Test 14. (\ion{H}{2} region with MHD).
    Left to right: slices of density at $t = 0.18, 0.53, 1.58$ Myr in
    the x-y plane.  The streamlines show the magnetic field.}
\end{figure}

\begin{figure}[t]
  \plotone{fig/test14_slices_yz}
  \caption{\label{fig:test14_2} Test 14. (\ion{H}{2} region with
    MHD). Slices of density (top) and the x-component of the magnetic
    field (bottom) in the y-z plane at $t = 0.18, 0.53, 1.58$ Myr
    (left to right).}
\end{figure}

Another prevelant physical component in astrophysics is a magnetic
field.  We utilize the new magnetohydrodynamics (MHD) framework
\citep{Wang09} in \enzo~v2.0 that uses an unsplit conservative
hydrodynamics solver and a hyperbolic $\nabla \cdot \mathbf{B} = 0$
cleaning method of \citet{Dedner02}.  This marriage of radiation
transport and MHD has already been demonstrated in \citet{Wang10}, but
for illustrative purposes, we show a test problem with an expanding
\ion{H}{2} region in an initially uniform density field and constant
magnetic field.  We use the same problem setup as
\citet{Krumholz07_ART} --- $\rho = 100~\cubecm$, $T = 11$ K, $L_{\rm
  box} = 20$ pc with a resolution of 256$^3$.  This ambient medium is
threaded by a magnetic field $\mathbf{B} = 14.2 \hat{\mathbf{x}} \;
\mu\mathrm{G}$.  The Alfv\'{e}n speed is 2.6 \kms.  The radiation
source is located in the center of the box with a luminosity $L = 4
\times 10^{46}$ ph s$^{-1}$ with a 17.6 eV mono-chromatic spectrum,
resulting in a Str\"{o}mgren radius $R_s = 0.5$ pc.  The simulation is
run for 1.58 Myr.  The hydrodynamics solver uses an HLL Riemann solver
\citep{HLL} and piecewise linear method (PLM) reconstruction
\citep{PLM} for the left and right states in this problem.

As the \ion{H}{2} region grows the magnetized medium, shown in Figures
\ref{fig:test14_1} and \ref{fig:test14_2}, it transforms from
spherical to oblate as it is magnetically confined in directions
perperdicular to the magnetic field.  This occurs at $t > 0.5$ Myr
because the magnetic pressure exceeds the thermal pressure, and the
gas can only flow along field lines.  \citeauthor{Krumholz07_ART}
observed some carbuncle artifacts along the ionization front; whereas
we see smooth density gradients, which is most likely caused by our
more diffusive choice of the HLL Riemann solver when compared to Roe's
Riemann solver used in \citet{Krumholz07_ART}, which also uses PLM as
a reconstruction method.  The evolution of the magnetic field lines
evolve in a similar manner as their results.

\section{Resolution Tests}

Resolution tests are important in validating the accuracy of the code
in most circumstances, especially in production simulations where the
initial environments surrounding radiation sources are unpredictable.
In this section, we show how our adaptive ray-tracing implementation
behaves when varying spatial, angular, frequency, and temporal
resolutions.

\subsection{Spatial resolution}

\begin{figure}[t]
  \epsscale{0.75}
  \plotone{fig/ifront_res}
  \caption{\label{fig:dx_dep1} Growth of the ionization front radius,
    compared to the analytical radius, in Test 1 with varying spatial
    resolutions.  At resolutions of $16^3$ and $32^3$, the ionization
    front is underestimated for the first $\sim25$ Myr but converges
    within 0.5\% of the higher resolution runs.}
  \epsscale{1}
\end{figure}

Here we use Test 1 (\S\ref{sec:test1}) as a testbed to investigate how
the evolution of the Str\"{o}mgren radius changes with resolution.  We
keep all aspects of the test the same, but use resolutions of 16$^3$,
32$^3$, 64$^3$, and 128$^3$.  In Figure \ref{fig:dx_dep1}, we show the
ratio $r_{\rm IF}/r_{\rm anyl}$, similar to Figure
\ref{fig:test1_ifront}, using these different resolutions.  The radii
in the $64^3$ and $128^3$ runs evolve almost identically.  Compared to
these resolutions, the lower $16^3$ and $32^3$ resolution runs only
lag behind by 1\% until 300 Myr, and afterwards it is larger by 0.5\%
than the higher resolution cases.  This shows that our method gives
accurate results, even in marginally resolved cases.

\subsection{Angular resolution}
\label{sec:ang_dep}

\begin{figure}[t]
  \epsscale{1}
  \plotone{fig/phic_slices}
  \caption{\label{fig:ang_dep1} Variations in the photo-ionization
    rates for different ray-to-cell samplings $\Phi_c$.  The colormap
    only spans a factor of 3 to enhance the contrast.  In comparision,
    the photo-ionization rate actually spans 4 orders of magnitude in
    this test.}
  \epsscale{1}
\end{figure}

\begin{figure}[t]
  \epsscale{0.75}
  \plotone{fig/phic_dep}
  \caption{\label{fig:ang_dep2} Standard deviations of the difference
    between the computed photo-ionization rates and an inverse square
    law as a function of ray-to-cell samplings $\Phi_c$ for different
    spatial resolutions.  There is no dependence on the spatial
    resolution, and the accuracy increases as $\sigma \propto
    \Phi_c^{-0.6}$.}
  \epsscale{1}
\end{figure}

The Cartesian grid must been sampled with sufficient rays in order to
calculate a smooth radiation field.  To determine the dependence on
angular resolution, we consider the propagation of radiation through
an optically thin, uniform medium.  The radiation field should follow
a $1/r^2$ profile.  As the grid is less sampled by rays, the deviation
from $1/r^2$ should increase.  This test is similar to Test 1, but the
medium has $\rho = 10^{-3}~\cubecm$, $T = 10^4$ K, and $1 - x_e =
10^{-4}$.  The radiation source is located in the center of a 6.6 kpc
domain.  The simulation is only run for one timestep because the
radiation field should be static in this optically-thin test.  

We consider minimum ray-to-cell ratios $\Phi_c = (1.1, 2.1, 3.1, 5.1,
10.1, 25.1)$.  Slices of the photo-ionization rates through the origin
are shown in Figure \ref{fig:ang_dep1} for these values of $\Phi_c$.
In this figure, we limit the colormap range to a factor of 3 to show
the nature of the artifacts in more contrast.  Unscaled, the rates in
the figures would span 4 orders of magnitude.  When $\Phi_c \le 3.1$,
the cell-to-cell variations are apparent because there are not enough
rays to sufficiently sample the radiation field, even with the
geometric correction factor $f_c$, whose improvements are shown later
in \S\ref{sec:test_fc}.  At $\Phi_c = 5.1$, these artifacts disappear,
leaving behind a shell artifact where the radiation fields do not
smoothly decrease as 1/$r^2$.  At higher values of $\Phi_c$, this
shell artifact vanishes as well.  

One measure of accuracy is the deviation from an 1/$r^2$ field because
this problem is optically-thin.  To depict the increase in accuracy
with ray sampling, we take the difference between the calculated
photo-ionization rate and a 1/$r^2$ field, and then plot the standard
deviation of this difference field versus angular resolution in Figure
\ref{fig:ang_dep2}.  We plot this relation for resolutions of $32^3$,
$64^3$, and $128^3$ and find no dependence on spatial resolution,
which is expected because we control the angular resolution in terms
of cell widths, not in absolute solid angles.  We find that the
deviation from an inverse square law decreases as $\sigma \propto
\Phi_c^{-0.6}$.

\subsection{Frequency resolution}

In practice, we find good agreement within 5--10\% of analytical
solutions in Tests 1, 2, and 5 with only one energy group.  We use
Test 5 (\S\ref{sec:test5}; an expanding \ion{H}{2} region with
hydrodynamics) to probe any differences in the solution when varying
the resolution of the spectrum.  In RT09, ZEUS-MP was used to
demonstrate the effect of a multi-frequency spectrum on the dynamics
of the ionization front in this test.  Instead of a single shock seen
in the mono-chromatic spectrum, a the shock obtained a double-peaked
structure in density and radial velocity.  We rerun Test 5 with a
$T=10^5$ K blackbody spectrum sampled by 1, 2, 4, 8, and 16 frequency
bins.

\subsection{Temporal resolution}
\label{sec:dt_dep}

The previous three dependencies did not affect the propagation of the
ionization front greatly.  However in our and others' past experience
\citep[e.g.][\bf{+others!}]{Petkova09}, the timestep, especially too
small of one, can drastically affect the ionization front velocity.
Here we use Test 1 but with $64^3$ resolution to compare different
timestepping methods --- restricted changes in \ion{H}{2}
(\S\ref{sec:dt_hi}), constant timesteps (\S\ref{sec:dt_const}), and
based on incident radiation (\S\ref{sec:dt_tau}).

\section{Methodology Tests}

Here we show tests that evaluate new features in Enzo+Moray, such as
the improvements from the geometric correction factor, optically-thin
approximations, treatment of X-ray radiation, and radiation pressure.
Lastly we test for any non-spherical artifacts in the case of two
sources.

\subsection{Improvements from the covering factor correction}
\label{sec:test_fc}

\begin{figure}[t]
  \epsscale{0.7}
  \plotone{fig/fc_slices}
  \caption{\label{fig:fc_slices} Slices of the photo-ionization rate
    in the x-y plane (top row) and x-z plane (bottom row) with (left
    column) and without (right column) the geometric correction.  The
    slices are through the origin.  In the x-y plane, it reduces the
    shell artifacts.  In the x-z plane, it reduces the severity of a
    non-spherical artifact delinated at a 45 degree angle, where the
    HEALPix scheme switches from polar to equatorial type pixels.}
  \epsscale{1}
\end{figure}

As discussed in \S\ref{sec:meth_fc}, non-spherical artifacts are
created by a mismatch between the HEALPix pixelization and the
Cartesian grid.  This is especially apparent in optically-thin
regions, where the area of the pixel is greater than the
$(1-e^{-\tau})$ absorption factor.  In this section, we repeat the
angular resolution tests in \S\ref{sec:ang_dep}.  Slices of the
photo-ionization rates through the origin are shown in Figure
\ref{fig:fc_slices}, depicting the improvements in spherical symmetry
and a closer agreement to a smooth $1/r^2$ profile.  Previous attempts
to reduce these artifacts either introduced a random rotation of the
HEALPix pixelization \citep[e.g.][]{Abel02_RT, Trac07, Krumholz07_ART}
or by increasing the ray-to-cell sampling.

In the x-y plane without the correction, there exists shell artifacts
where the photo-ionization rates abruptly drops when the rays are
split.  This occurs because the photon flux in the rays are constant,
so \kph~is purely dependent on the ray segment length through each
cell.  Geometric dilution mainly occurs when the number of rays
passing through a cell decreases.  With the correction, geometric
dilution also occurs when the ray's solid angle only partially covers
the cell.  This by itself alleviates these shell artifacts.  In the
x-z plane without the correction, there is a non-spherical artifact
delinated at a 45 degree angle.  In the lower region, the rays are
assoicated with equatorial HEALPix pixels, and in the upper region,
they are polar HEALPix pixels.  This artifact is not seen in the x-y
plane because all rays are of a equatorial type.  The geometric
correction smooths this artifact but does not completely remove it.

\subsection{Optically-thin approximation}

In practice, we have found it difficult to transition from this
optically-thin approximation to the optically-thick regime without
producing artifacts in the photo-ionization rate \kph.  The artifacts
exist in cells that intersect the $\tau_{\rm thin} \equiv 0.1$
surface.  These cells are split into the optically thin and thick
definitions, and we have not determined a good technique to avoid any
artifacts.  We use the optically thin problem used in the angular
resolution test (\S\ref{sec:ang_dep}) to show these artifacts in \kph.

\subsection{X-Ray secondary ionizations and reduced photo-heating}

\begin{figure}[t]
  \epsscale{0.6}
  \plotone{fig/xray_profiles.eps}
  \caption{\label{fig:xray_fig} Radial profiles of temperature and
    ionized fraction showing the effects of secondary ionizations from
    a mono-chromatic 1 keV spectrum.  The high energy photons can
    ionize multiple hydrogen atoms, increasing the ionized fraction.
    In part, less radiation goes into thermal energy, lowering the
    temperature.}
  \epsscale{1}
\end{figure}

Here we test our implementation of secondary ionizations from
high-energy photons above 100 eV, described in \S\ref{sec:xrays} and
used in \citet{Alvarez09} in the context of accreting black holes.  We
use the same setup as Test 5 but with an increased luminosity $L =
10^{50}$ erg s$^{-1}$ and a mono-chromatic spectrum of 1 keV.  Figure
\ref{fig:xray_fig} compares the density, temperature, ionized
fraction, and neutral fraction of the expanding \ion{H}{2} region
considering secondary ionizations and reduced photo-heating and
considering only one ionization per photon and the remaining energy
being thermalized.

\subsection{Radiation pressure}

\begin{figure}[t]
  \plottwo{fig/noRP_profiles.eps}{fig/RP_profiles.eps}
  \caption{\label{fig:rp_profile150} (a) No radiation pressure.
    Radial profiles of (clockwise from top left) density, temperature,
    radial velocity, and neutral fraction.  Time units in the legned
    are in kyr.  (b) Radial profiles with radiation pressure.  The
    momentum transferred to the gas drives out the gas at higher
    velocities than without radiation pressure.  Afterwards the
    central region is underpressurized, and the gas infalls toward the
    center, as seen at t = 60 kyr.  Then the radiation pressure
    continues to force the gas outwards, increasing gas velocites up
    to 50 km/s.}
\end{figure}

Radiation pressure affects gas dynamics in an \ion{H}{2} region when
its force is comparable to the acceleration created by gas pressure of
the heated region.  The imparted acceleration on a hydrogen atom
$\mathbf{a}_{\rm rp} = E_{\rm ph}/c$.  This is especially important
when the ionization front is in its initial R-type phase.  Thus we
construct a test that focuses on a small scales, compared to the
Str\"{o}mgren radius.  The domain has a size of 8 pc with a uniform
density $\rho = 2900 \cubecm$ and initial temperature $T = 10^3$ K.
The source is located at the origin with a luminosity $L = 10^{50}$ ph
s$^{-1}$ and a $T=10^5$ K blackbody spectrum.  We use one energy group
$E_{\rm ph} = 29.6$ eV.  The grid is adaptively refined on overdensity
with the same criteron as Test 8.  The simulation is run for 150 kyr.

\subsection{Consolidated \ion{H}{2} region with two sources}

Here we test for any inaccuracies in the case of multiple sources.  We
use the same test problem as \citet[][\S5.1.2]{Petkova09}, which has
two sources with luminosities of $5 \times 10^{48}$ ph s$^{-1}$ and
are separated by 8 kpc.  The ambient medium is static with a uniform
density of $10^{-3}$ \cubecm and $T = 10^4$ K.  This setup is similar
to Test 1.  The domain has a resolution of $128^3$ and is 20 kpc in
width, and the problem is run for 500 Myr.  

The \ion{H}{2} regions grow to $r = 4$ kpc where they overlap.  Then
the two sources are enveloped in a common, elongated \ion{H}{2}
region.

\section{Parallel Performance}

Last we demonstrate the parallel performance of Enzo+Moray in weak and
strong scaling tests.  For large simulations to consider radiative
transfer, it is imperative that the code scales to large number of
processors.

\subsection{Weak Scaling}
\label{sec:weak_sc}

Weak scaling tests demonstrates how the code scales with the number of
processors with a constant amount of work per processor.  Here we
construct a test problem with a $64^3$ block per processor.  The grid
is not adaptively refined.  The physical setup of the problem is the
same as Test 5 with a uniform density $\rho = 10^{-3} \cubecm$ and
initial temperature $T = 100$ K.  Each block has the same size of 15
kpc as Test 5.  At the center of each grid, there exists a radiation
source with a luminosity $L = 5 \times 10^{48}$ ph s$^{-1}$ and a 17
eV mono-chromatic spectrum.  The problem is run for 100 Myr.  We run
this test with $N_p = 2^n$ processors with $n = [0,1 \dots 11,12]$.
The domain has $(N_x, N_y, N_z)$ blocks that is determined with the
MPI routine \texttt{MPI\_Dims\_create}.  For example with $n = 7$, the
problem is decomposed into $(N_x, N_y, N_z) = (4,4,8)$ blocks,
producing a $256 \times 256 \times 512$ grid.

\subsection{Strong Scaling}
\label{sec:strong_sc}

Strong scaling tests shows how the problem scales with the number of
processors for the same problem.  The overhead associated with the
structured AMR framework in \enzo~can limit the strong scalibility.
One key property of strong scaling is that each processor must have
sufficient work to compute, compared to the communication involved.
Here we use a small-box reionization calculation with $L_{\rm box} =
3$ Mpc/h, a resolution of $256^3$, and 6 levels of refinement.  We
measure the time spent on the hydrodynamics, non-equilibrium
chemistry, the rebuilding of the AMR hierarhcy, and ray tracing in a
single level-0 timestep, lasting 5 Myr, at $z=8$.  There are ??? point
sources in this calculation at this redshift.

\section{Summary}

\acknowledgments

J.H.W. is supported by the Hubble Fellowship etc.  The majority of the
analysis and plots were done with \texttt{yt} \citep{yt}.

\clearpage
%\input biblio.tex
\bibliography{ms}

\end{document}
